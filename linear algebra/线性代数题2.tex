\documentclass[a4paper]{article}
\usepackage{amsmath,amsfonts,amsthm,amssymb}
\usepackage{bm}
\usepackage{draftwatermark}
\SetWatermarkText{http://blog.sciencenet.cn/u/Yaleking}%设置水印文字
\SetWatermarkLightness{0.8}%设置水印亮度
\SetWatermarkScale{0.35}%设置水印大小
\usepackage{hyperref}
\usepackage{geometry}
\usepackage{yhmath}
\usepackage{pstricks-add}
\usepackage{framed,mdframed}
\usepackage{graphicx,color} 
\usepackage{mathrsfs,xcolor} 
\usepackage[all]{xy}
\usepackage{fancybox} 
\usepackage{xeCJK}
\newtheorem*{theo}{Theorem}
\newtheorem*{exe}{Exercise}
\newenvironment{theorem}
{\bigskip\begin{mdframed}\begin{theo}}
    {\end{theo}\end{mdframed}\bigskip}
\newenvironment{exercise}
{\bigskip\begin{mdframed}\begin{exe}}
    {\end{exe}\end{mdframed}\bigskip}
\geometry{left=2.5cm,right=2.5cm,top=2.5cm,bottom=2.5cm}
\setCJKmainfont[BoldFont=SimHei]{SimSun}
\newcommand{\D}{\displaystyle}\newcommand{\ri}{\Rightarrow}
\newcommand{\ds}{\displaystyle} \renewcommand{\ni}{\noindent}
\newcommand{\pa}{\partial} \newcommand{\Om}{\Omega}
\newcommand{\om}{\omega} \newcommand{\sik}{\sum_{i=1}^k}
\newcommand{\vov}{\Vert\omega\Vert} \newcommand{\Umy}{U_{\mu_i,y^i}}
\newcommand{\lamns}{\lambda_n^{^{\scriptstyle\sigma}}}
\newcommand{\chiomn}{\chi_{_{\Omega_n}}}
\newcommand{\ullim}{\underline{\lim}} \newcommand{\bsy}{\boldsymbol}
\newcommand{\mvb}{\mathversion{bold}} \newcommand{\la}{\lambda}
\newcommand{\La}{\Lambda} \newcommand{\va}{\varepsilon}
\newcommand{\be}{\beta} \newcommand{\al}{\alpha}
\newcommand{\dis}{\displaystyle} \newcommand{\R}{{\mathbb R}}
\newcommand{\N}{{\mathbb N}} \newcommand{\cF}{{\mathcal F}}
\newcommand{\gB}{{\mathfrak B}} \newcommand{\eps}{\epsilon}
\renewcommand\refname{参考文献}\renewcommand\figurename{图}
\usepackage[]{caption2} 
\renewcommand{\captionlabeldelim}{}
\begin{document}
\title{\huge{\bf{Linear algebra,Exercise 2}}} \author{\small{叶卢
    庆\footnote{Luqing
      Ye(1992---),E-mail:yeluqingmathematics@gmail.com}}}
\maketitle
\begin{exercise}\footnote{This exercise is from
    www.math.ucla.edu/~tao/resource/general/115a.3.02f/practice.pdf}
  Let $T:V\to W$ be a linear transformation from one vector space $V$
  to another vector space $W$.Let $v_1,\cdots,v_n$ be vectors in
  $V$.Assume the span of $v_1,\cdots,v_n$ contains the null space
  $N(T)$ of $T$,and assume that the vectors $T(v_1),\cdots,T(v_n)$
  span $W$.Prove that the vectors $v_1,\cdots,v_n$ span $V$.
\end{exercise}
\begin{proof}
Prove by contradiction.Otherwise,there exists $v_{n+1}\in V$,such that
$v_{n+1}\not\in Span(v_1,\cdots,v_n)$.From $Span
(T(v_1),\cdots,T(v_n))=W$,we have
$$
T(v_{n+1})=a_1T(v_1)+\cdots+a_nT(v_n),
$$
so 
$$
T(v_{n+1}-a_1v_1-\cdots-a_nv_n)=0_w,
$$
which means that $v_{n+1}-a_1v_1-\cdots-a_nv_n\in N(T)$.From
$N(T)\subset Span (v_1,\cdots,v_n)$,we have
$$
v_{n+1}-a_1v_1-\cdots-a_nv_n=b_1v_1+\cdots+b_nv_n,
$$
so 
$$
v_{n+1}=(a_1+b_1)v_1+\cdots+(a_n+b_n)v_n.
$$
This contradicts $v_{n+1}\not\in Span (v_1,\cdots,v_n)$.
\end{proof}
\end{document}
