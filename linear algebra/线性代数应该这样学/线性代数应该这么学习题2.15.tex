\documentclass[a4paper]{article}
\usepackage{amsmath,amsfonts,amsthm,amssymb}
\usepackage{bm}
\usepackage{draftwatermark}
\SetWatermarkText{http://blog.sciencenet.cn/u/Yaleking}%设置水印文字
\SetWatermarkLightness{0.8}%设置水印亮度
\SetWatermarkScale{0.35}%设置水印大小
\usepackage{hyperref}
\usepackage{geometry}
\usepackage{yhmath}
\usepackage{pstricks-add}
\usepackage{framed,mdframed}
\usepackage{graphicx,color} 
\usepackage{mathrsfs,xcolor} 
\usepackage[all]{xy}
\usepackage{fancybox} 
\usepackage{xeCJK}
\newtheorem{theo}{定理}
\newtheorem*{exe}{题目}
\newtheorem*{rem}{评论}
\newmdtheoremenv{lemma}{引理}
\newmdtheoremenv{corollary}{推论}
\newmdtheoremenv{example}{例}
\newenvironment{theorem}
{\bigskip\begin{mdframed}\begin{theo}}
    {\end{theo}\end{mdframed}\bigskip}
\newenvironment{exercise}
{\bigskip\begin{mdframed}\begin{exe}}
    {\end{exe}\end{mdframed}\bigskip}
\geometry{left=2.5cm,right=2.5cm,top=2.5cm,bottom=2.5cm}
\setCJKmainfont[BoldFont=SimHei]{SimSun}
\renewcommand{\today}{\number\year 年 \number\month 月 \number\day 日}
\newcommand{\D}{\displaystyle}\newcommand{\ri}{\Rightarrow}
\newcommand{\ds}{\displaystyle} \renewcommand{\ni}{\noindent}
\newcommand{\pa}{\partial} \newcommand{\Om}{\Omega}
\newcommand{\om}{\omega} \newcommand{\sik}{\sum_{i=1}^k}
\newcommand{\vov}{\Vert\omega\Vert} \newcommand{\Umy}{U_{\mu_i,y^i}}
\newcommand{\lamns}{\lambda_n^{^{\scriptstyle\sigma}}}
\newcommand{\chiomn}{\chi_{_{\Omega_n}}}
\newcommand{\ullim}{\underline{\lim}} \newcommand{\bsy}{\boldsymbol}
\newcommand{\mvb}{\mathversion{bold}} \newcommand{\la}{\lambda}
\newcommand{\La}{\Lambda} \newcommand{\va}{\varepsilon}
\newcommand{\be}{\beta} \newcommand{\al}{\alpha}
\newcommand{\dis}{\displaystyle} \newcommand{\R}{{\mathbb R}}
\newcommand{\N}{{\mathbb N}} \newcommand{\cF}{{\mathcal F}}
\newcommand{\gB}{{\mathfrak B}} \newcommand{\eps}{\epsilon}
\renewcommand\refname{参考文献}\renewcommand\figurename{图}
\usepackage[]{caption2} 
\renewcommand{\captionlabeldelim}{}
\begin{document}
\title{\huge{\bf{线性代数应该这么学,习题2.15}}} \author{\small{叶卢
    庆\footnote{叶卢庆(1992---),男,杭州师范大学理学院数学与应用数学专业
      本科在读,E-mail:yeluqingmathematics@gmail.com}}}
\maketitle
\begin{exercise}[线性代数应该这么学,习题2.15]
  如果 $U_1,U_2,U_3$ 是有限维向量空间的子空间,是否有
  \begin{align*}
    \dim (U_1+U_2+U_3)&=\dim U_1+\dim U_2+\dim U_3\\&-\dim(U_1\cap
    U_2)-\dim (U_1\cap U_3)\\&-\dim(U_2\cap U_3)\\&+\dim(U_1\cap U_2\cap U_3).
  \end{align*}
\end{exercise}
\begin{proof}[\textbf{解}]
我们尝试对 $U_3$ 的维数作归纳.当 $\dim U_3=0$,也就是 $U_3=\{0\}$ 时,题
目中欲证明的式子会化成
\begin{align*}
  \dim(U_1+U_2)=\dim U_1+\dim U_2-\dim (U_1\cap U_2),
\end{align*}
这正是
\href{http://blog.sciencenet.cn/home.php?mod=space&uid=604208&do=blog&id=820615}{维数定理}.假设当 $\dim U_3=k(k\geq 0)$ 时,题目中的式子成立.则当 $\dim
U_3=k+1$ 时,设$U_3$的一组基为$\{v_1,\cdots,v_k,v_{k+1}\}$.
\begin{itemize}
\item 如果$v_{k+1}\not\in U_1+U_2$,则我们有
$$
\begin{cases}
  \dim (U_1+U_2+U_3)=\dim (U_1+U_2+Span\{v_1,\cdots,v_k\})+1,\\
\dim U_1=\dim U_1,\\
\dim U_2=\dim U_2,\\
\dim U_3=\dim(Span \{v_1,\cdots,v_k\})+1,\\
\dim (U_1\cap U_2)=\dim (U_1\cap U_2),\\
\dim (U_1\cap U_3)=\dim (U_1\cap Span\{v_1,\cdots,v_k\}),\\
\dim (U_2\cap U_3)=\dim (U_2\cap Span\{v_1,\cdots,v_k\}),\\
\dim (U_1\cap U_2\cap U_3)=\dim (U_1\cap U_2\cap Span\{v_1,\cdots,v_k\}).
\end{cases}
$$
且根据归纳假设,
\begin{align*}
    \dim (U_1+U_2+Span\{v_1,\cdots,v_k\})&=\dim U_1+\dim U_2+\dim Span\{v_1,\cdots,v_k\}\\&-\dim(U_1\cap
    U_2)-\dim (U_1\cap Span\{v_1,\cdots,v_k\})\\&-\dim(U_2\cap Span\{v_1,\cdots,v_k\})\\&+\dim(U_1\cap U_2\cap Span\{v_1,\cdots,v_k\}).
\end{align*}
综合以上式子,可得当$\dim U_3=k+1$ 时,题目中的式子仍然成立.
\item 当$v_{k+1}\in U_1+U_2$,且 $v_{k+1}\in U_1$或$v_{k+1}\in
  U_2$ 时,易得题目中的式子也成立.
\item 当$v_{k+1}\in U_1+U_2$,且$v_{k+1}\not\in U_1\bigcup U_2$ 时,题目
  中的式子就不成立了.
\end{itemize}
可见递推失败.也就是说,题目中的式子是个错误的式子.
\end{proof}
\end{document}
