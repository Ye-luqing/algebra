\documentclass[a4paper]{article}
\usepackage{amsmath,amsfonts,amsthm,amssymb} \usepackage{bm}
\usepackage{draftwatermark}
\SetWatermarkText{http://blog.sciencenet.cn/u/Yaleking}%设置水印文字
\SetWatermarkLightness{0.8}%设置水印亮度
\SetWatermarkScale{0.35}%设置水印大小
\usepackage{hyperref} \usepackage{geometry} \usepackage{yhmath}
\usepackage{pstricks-add} \usepackage{framed,mdframed}
\usepackage{graphicx,color} \usepackage{mathrsfs,xcolor}
\usepackage[all]{xy} \usepackage{fancybox} \usepackage{xeCJK}
\newtheorem*{theo}{定理} 
\newtheorem*{exa}{习题}
\newtheorem*{rem}{评论}
\newenvironment{theorem}
{\bigskip\begin{mdframed}\begin{theo}}
    {\end{theo}\end{mdframed}\bigskip} 
\newenvironment{remark}
{\bigskip\begin{mdframed}\begin{rem}}
    {\end{rem}\end{mdframed}\bigskip} 
\newenvironment{example}
{\bigskip\begin{mdframed}\begin{exa}}
    {\end{exa}\end{mdframed}\bigskip}
\geometry{left=2.5cm,right=2.5cm,top=2.5cm,bottom=2.5cm}
\setCJKmainfont[BoldFont=SimHei]{SimSun}
\newcommand{\D}{\displaystyle}\newcommand{\ri}{\Rightarrow}
\newcommand{\ds}{\displaystyle} \renewcommand{\ni}{\noindent}
\newcommand{\pa}{\partial} \newcommand{\Om}{\Omega}
\newcommand{\om}{\omega} \newcommand{\sik}{\sum_{i=1}^k}
\newcommand{\vov}{\Vert\omega\Vert} \newcommand{\Umy}{U_{\mu_i,y^i}}
\newcommand{\lamns}{\lambda_n^{^{\scriptstyle\sigma}}}
\newcommand{\chiomn}{\chi_{_{\Omega_n}}}
\newcommand{\ullim}{\underline{\lim}} \newcommand{\bsy}{\boldsymbol}
\newcommand{\mvb}{\mathversion{bold}} \newcommand{\la}{\lambda}
\newcommand{\La}{\Lambda} \newcommand{\va}{varepsilon}
\newcommand{\be}{\beta} \newcommand{\al}{\alpha}
\newcommand{\dis}{\displaystyle} \newcommand{\R}{{\mathbb R}}
\renewcommand{\today}{\number\year 年 \number\month 月 \number\day 日}
\newcommand{\N}{{\mathbb N}} \newcommand{\cF}{{\mathcal F}}
\newcommand{\gB}{{\mathfrak B}} \newcommand{\eps}{\epsilon}\newcommand{\op}{\operatorname}
\renewcommand\refname{参考文献}\renewcommand\figurename{图}
\usepackage[]{caption2} \renewcommand{\captionlabeldelim}{}
\begin{document}
\title{\huge{\bf{线性代数应该这样学,习题5.14}}} \author{\small{叶卢
    庆\footnote{叶卢庆(1992---),男,杭州师范大学理学院数学与应用数学专业
      本科在读,E-mail:yeluqingmathematics@gmail.com}}}
\maketitle\ni
\begin{example}
  设$S,T\in \mathcal{L}(V)$,并且$S$是可逆的.证明:若$p\in
  \mathcal{P}(\mathbf{F})$是多项式,则
$$
p(S^{-1}TS)=Sp(T)S^{-1}.
$$
\end{example}
\begin{proof}[\textbf{评论}]
  这个题目的证明是很容易的.我只想谈谈其背后的意义.显然,矩阵和线性变换是
  有区别的.矩阵,其实是给定了有序基之后,线性变换的具体表示,表示的是有序
  基里各个向量之间的线性关系.如果没有给定有序基,那么矩阵就是没有意义的.而
  线性变换是独立于基的,无论有没有选定一组有序基,线性变换始终在那里——当
  然,如果没有选定有序基的话,你想要描述一个线性变换将会变得相当困难.选定
  了有序基之后,描述线性变换就简单了.一般来说,在不同有序基下,同一个线性
  变换$T$会有不同的矩阵表示,但是这些不同的矩阵都是相似的(similar).考虑
  了在不同有序基下的同一个线性变换后,我们再来考虑在同一个有序基下的不同
  线性变换.如果在同一个有序基$\alpha$下的不同线性变换$T_1,T_2$所对应的
  矩阵是相似的,这意味着什么呢?这意味着,存在一组基$\beta$,换$\beta$来看$T_1$,将会看
  到在$\alpha$下看$T_2$同样的效果.
\end{proof}
\end{document}





























