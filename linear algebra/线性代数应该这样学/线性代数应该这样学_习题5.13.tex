\documentclass[a4paper]{article}
\usepackage{amsmath,amsfonts,amsthm,amssymb} \usepackage{bm}
\usepackage{draftwatermark}
\SetWatermarkText{http://blog.sciencenet.cn/u/Yaleking}%设置水印文字
\SetWatermarkLightness{0.8}%设置水印亮度
\SetWatermarkScale{0.35}%设置水印大小
\usepackage{hyperref} \usepackage{geometry} \usepackage{yhmath}
\usepackage{pstricks-add} \usepackage{framed,mdframed}
\usepackage{graphicx,color} \usepackage{mathrsfs,xcolor}
\usepackage[all]{xy} \usepackage{fancybox} \usepackage{xeCJK}
\newtheorem*{theo}{定理} 
\newtheorem*{exa}{习题}
\newtheorem*{rem}{评论}
\newenvironment{theorem}
{\bigskip\begin{mdframed}\begin{theo}}
    {\end{theo}\end{mdframed}\bigskip} 
\newenvironment{remark}
{\bigskip\begin{mdframed}\begin{rem}}
    {\end{rem}\end{mdframed}\bigskip} 
\newenvironment{example}
{\bigskip\begin{mdframed}\begin{exa}}
    {\end{exa}\end{mdframed}\bigskip}
\geometry{left=2.5cm,right=2.5cm,top=2.5cm,bottom=2.5cm}
\setCJKmainfont[BoldFont=SimHei]{SimSun}
\newcommand{\D}{\displaystyle}\newcommand{\ri}{\Rightarrow}
\newcommand{\ds}{\displaystyle} \renewcommand{\ni}{\noindent}
\newcommand{\pa}{\partial} \newcommand{\Om}{\Omega}
\newcommand{\om}{\omega} \newcommand{\sik}{\sum_{i=1}^k}
\newcommand{\vov}{\Vert\omega\Vert} \newcommand{\Umy}{U_{\mu_i,y^i}}
\newcommand{\lamns}{\lambda_n^{^{\scriptstyle\sigma}}}
\newcommand{\chiomn}{\chi_{_{\Omega_n}}}
\newcommand{\ullim}{\underline{\lim}} \newcommand{\bsy}{\boldsymbol}
\newcommand{\mvb}{\mathversion{bold}} \newcommand{\la}{\lambda}
\newcommand{\La}{\Lambda} \newcommand{\va}{varepsilon}
\newcommand{\be}{\beta} \newcommand{\al}{\alpha}
\newcommand{\dis}{\displaystyle} \newcommand{\R}{{\mathbb R}}
\renewcommand{\today}{\number\year 年 \number\month 月 \number\day 日}
\newcommand{\N}{{\mathbb N}} \newcommand{\cF}{{\mathcal F}}
\newcommand{\gB}{{\mathfrak B}} \newcommand{\eps}{\epsilon}\newcommand{\op}{\operatorname}
\renewcommand\refname{参考文献}\renewcommand\figurename{图}
\usepackage[]{caption2} \renewcommand{\captionlabeldelim}{}
\begin{document}
\title{\huge{\bf{线性代数应该这样学,习题5.13}}}
\author{\small{叶卢庆\footnote{叶卢庆(1992---),男,杭州师范大学理学院数
      学与应用数学专业本科在读,E-mail:yeluqingmathematics@gmail.com}}}
\maketitle\ni
\begin{example}
设 $T\in \mathcal{L}(V)$,并且$V$的每个$\dim V-1$维子空间在$T$下都不变.证
明$T$是恒等算子的标量倍.
\end{example}
\begin{proof}[\textbf{证明}]
设$\alpha=(v_1,\cdots,v_n)$是$V$的一组有序基.设向量$v_i$张成$V$的一维
线性子空间$U_i$.则
$$
V=U_1\oplus\cdots\oplus U_n.
$$
假若存在 $1\leq j\leq n$,使得
$$
T(v_j)\neq \lambda v_j,
$$
则 $T(v_j)$ 必定可以分解为 $w+\lambda' v_j$,其中$\lambda'\in
\mathbf{F}$,$w$ 可以被 $\{v_1,\cdots,v_n\}\backslash \{v_j\}$ 中的向量
进行唯一的线性表示.由于$w\neq 0$,因此必定存在$k\neq j$,使得 $w$ 被
$\{v_1,\cdots,v_n\}\backslash \{v_j\}$中的向量线性表示时,$v_k$ 前面的
系数 $a_k\neq 0$.现在我们看基$\{v_1,\cdots,v_n\}\backslash \{v_k\}$张
成的$\dim V-1$维线性子空间,显然这个线性子空间并非$T$的不变子空间.于是
假设错误,于是$\forall 1\leq j\leq n$,$T(v_j)=\lambda_j v_j $,其中
$\lambda_j\in \mathbf{F}$.
\end{proof}
\end{document}





























