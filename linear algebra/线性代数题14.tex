\documentclass[a4paper]{article}
\usepackage{amsmath,amsfonts,amsthm,amssymb}
\usepackage{bm}
\usepackage{draftwatermark}
\SetWatermarkText{http://blog.sciencenet.cn/u/Yaleking}%设置水印文字
\SetWatermarkLightness{0.8}%设置水印亮度
\SetWatermarkScale{0.35}%设置水印大小
\usepackage{hyperref}
\usepackage{geometry}
\usepackage{yhmath}
\usepackage{pstricks-add}
\usepackage{framed,mdframed}
\usepackage{graphicx,color} 
\usepackage{mathrsfs,xcolor} 
\usepackage[all]{xy}
\usepackage{fancybox} 
\usepackage{xeCJK}
\newtheorem{theo}{定理}
\newtheorem*{exe}{题目}
\newtheorem*{rem}{评论}
\newmdtheoremenv{lemma}{引理}
\newmdtheoremenv{corollary}{推论}
\newmdtheoremenv{example}{例}
\newenvironment{theorem}
{\bigskip\begin{mdframed}\begin{theo}}
    {\end{theo}\end{mdframed}\bigskip}
\newenvironment{exercise}
{\bigskip\begin{mdframed}\begin{exe}}
    {\end{exe}\end{mdframed}\bigskip}
\geometry{left=2.5cm,right=2.5cm,top=2.5cm,bottom=2.5cm}
\setCJKmainfont[BoldFont=SimHei]{SimSun}
\renewcommand{\today}{\number\year 年 \number\month 月 \number\day 日}
\newcommand{\D}{\displaystyle}\newcommand{\ri}{\Rightarrow}
\newcommand{\ds}{\displaystyle} \renewcommand{\ni}{\noindent}
\newcommand{\pa}{\partial} \newcommand{\Om}{\Omega}
\newcommand{\om}{\omega} \newcommand{\sik}{\sum_{i=1}^k}
\newcommand{\vov}{\Vert\omega\Vert} \newcommand{\Umy}{U_{\mu_i,y^i}}
\newcommand{\lamns}{\lambda_n^{^{\scriptstyle\sigma}}}
\newcommand{\chiomn}{\chi_{_{\Omega_n}}}
\newcommand{\ullim}{\underline{\lim}} \newcommand{\bsy}{\boldsymbol}
\newcommand{\mvb}{\mathversion{bold}} \newcommand{\la}{\lambda}
\newcommand{\La}{\Lambda} \newcommand{\va}{\varepsilon}
\newcommand{\be}{\beta} \newcommand{\al}{\alpha}
\newcommand{\dis}{\displaystyle} \newcommand{\R}{{\mathbb R}}
\newcommand{\N}{{\mathbb N}} \newcommand{\cF}{{\mathcal F}}
\newcommand{\gB}{{\mathfrak B}} \newcommand{\eps}{\epsilon}
\renewcommand\refname{参考文献}\renewcommand\figurename{图}
\usepackage[]{caption2} 
\renewcommand{\captionlabeldelim}{}
\begin{document}
\title{\huge{\bf{线性代数,习题14}}} \author{\small{叶卢
    庆\footnote{叶卢庆(1992---),男,杭州师范大学理学院数学与应用数学专业
      本科在读,E-mail:yeluqingmathematics@gmail.com}}}
\maketitle
\begin{exercise}\footnote{来自 http://www.math.ucla.edu/~tao/resource/general/115a.3.02f/final.pdf}
  线性映射 $T:\mathbf{R}^4\to \mathbf{R}^3$ 定义为
$$
T(x,y,z,w):=(x+y+z,y+2z+3w,x-z-2w).
$$
\begin{itemize}
\item 求 $\dim(\ker T)$.
\item 求 $\ker T$ 的一组基.
\item 求 $T$ 的值域的一组基.
\end{itemize}
\end{exercise}
\begin{proof}[\textbf{证明}]
\begin{itemize}
\item $T$ 对应的矩阵为
$$
\begin{pmatrix}
  1&1&1&0\\
0&1&2&3\\
1&0&-1&-2
\end{pmatrix}.
$$
我们将该矩阵进行初等行变换,
\begin{align*}
  \begin{pmatrix}
  1&1&1&0\\
0&1&2&3\\
1&0&-1&-2
\end{pmatrix}\ri
\begin{pmatrix}
  1&1&1&0\\
0&1&2&3\\
0&-1&-2&-2
\end{pmatrix}\ri
\begin{pmatrix}
  1&1&1&0\\
0&1&2&3\\
0&0&0&1
\end{pmatrix}.
\end{align*}
因此 $\dim (\ker T)=1$.
\item $\{(1,-2,1,0)\}$.
\item $\{(1,0,0),(1,1,0),(0,3,1)\}$.
\end{itemize}
\end{proof}
\end{document}





























