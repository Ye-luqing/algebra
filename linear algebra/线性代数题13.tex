\documentclass[a4paper]{article}
\usepackage{amsmath,amsfonts,amsthm,amssymb} \usepackage{bm}
\usepackage{draftwatermark,euler,epigraph}
\setlength{\epigraphwidth}{0.8\textwidth}
\SetWatermarkText{http://blog.sciencenet.cn/u/Yaleking}%设置水印文字
\SetWatermarkLightness{0.8}%设置水印亮度
\SetWatermarkScale{0.35}%设置水印大小
\usepackage{hyperref} \usepackage{geometry} \usepackage{yhmath}
\usepackage{pstricks-add} \usepackage{framed,mdframed}
\usepackage{graphicx,color} \usepackage{mathrsfs,xcolor}
\usepackage[all]{xy} \usepackage{fancybox} \usepackage{xeCJK}
\newtheorem*{theo}{定理} \newtheorem*{exe}{习题} \newenvironment{theorem}
{\bigskip\begin{mdframed}\begin{theo}}
    {\end{theo}\end{mdframed}\bigskip} \newenvironment{exercise}
{\bigskip\begin{mdframed}\begin{exe}}
    {\end{exe}\end{mdframed}\bigskip}
\geometry{left=2.5cm,right=2.5cm,top=2.5cm,bottom=2.5cm}
\setCJKmainfont[BoldFont=SimHei]{SimSun}
\newcommand{\D}{\displaystyle}\newcommand{\ri}{\Rightarrow}
\newcommand{\ds}{\displaystyle} \renewcommand{\ni}{\noindent}
\newcommand{\pa}{\partial} \newcommand{\Om}{\Omega}
\newcommand{\om}{\omega} \newcommand{\sik}{\sum_{i=1}^k}
\newcommand{\vov}{\Vert\omega\Vert} \newcommand{\Umy}{U_{\mu_i,y^i}}
\newcommand{\lamns}{\lambda_n^{^{\scriptstyle\sigma}}}
\newcommand{\chiomn}{\chi_{_{\Omega_n}}}
\newcommand{\ullim}{\underline{\lim}} \newcommand{\bsy}{\boldsymbol}
\newcommand{\mvb}{\mathversion{bold}} \newcommand{\la}{\lambda}
\newcommand{\La}{\Lambda} \newcommand{\va}{\varepsilon}
\newcommand{\be}{\beta} \newcommand{\al}{\alpha}
\newcommand{\dis}{\displaystyle} \newcommand{\R}{{\mathbb R}}
\renewcommand{\today}{\number\year 年 \number\month 月 \number\day 日}
\newcommand{\N}{{\mathbb N}} \newcommand{\cF}{{\mathcal F}}
\newcommand{\gB}{{\mathfrak B}} \newcommand{\eps}{\epsilon}
\renewcommand\refname{参考文献}\renewcommand\figurename{图}
\usepackage[]{caption2} \renewcommand{\captionlabeldelim}{}
\begin{document}
\title{\huge{\bf{线性代数,习题 13}}} \author{\small{叶卢庆\footnote{叶
      卢庆(1992---),E-mail:yeluqingmathematics@gmail.com}}}
\maketitle\ni
\begin{exercise}
记 $P_3(\mathbf{R})$ 是所有次数不超过 $3$ 的实系数多项式的集合.设
$T:P_3(\mathbf{R})\to P_3(\mathbf{R})$ 是线性变换
$$
Tf:=\frac{df}{dx}.
$$
设 $\beta:=(1,x,x^2,x^3)$ 是 $P_3(\mathbf{R})$ 的一组基.
\begin{itemize}
\item 计算 $[T]_{\beta}^{\beta}$.
\item 计算矩阵 $[T]_{\beta}^{\beta}$ 的特征多项式.
\item 求 $T$ 的特征值和特征向量.
\item $T$ 是否是可对角化的?
\end{itemize}
\end{exercise}
\begin{proof}[\textbf{证明}]
  \begin{itemize}
  \item 设 $\beta=(v_1,v_2,v_{3},v_4)$,则
    $T(1)=0=0v_1+0v_2+0v_3+0v_4$,$T(x)=1=1v_1+0v_2+0v_3+0v_4$,$T(x^2)=2x=0v_1+2v_2+0v_3+0v_4$,$T(x^3)=3x^2=0v_1+0v_2+3v_3+0v_4$,
    因此
$$
[T]_{\beta}^{\beta}=
\begin{pmatrix}
  0&1&0&0\\
0&0&2&0\\
0&0&0&3\\
0&0&0&0
\end{pmatrix}.
$$
\item 
  \begin{align*}
    \begin{vmatrix}
      -\lambda&1&0&0\\
0&-\lambda&2&0\\
0&0&-\lambda&3\\
0&0&0&-\lambda
    \end{vmatrix}=\lambda^4.
  \end{align*}
\item $T$ 所对应的矩阵为
$$
\begin{pmatrix}
0&0&0&0\\
1&0&0&0\\
0&2&0&0\\
0&0&3&0
\end{pmatrix}.
$$
因此 $T$ 的特征多项式为
$$
\begin{vmatrix}
  -\lambda&0&0&0\\
1&-\lambda&0&0\\
0&2&-\lambda&0\\
0&0&3&-\lambda
\end{vmatrix}=\lambda^4.
$$
特征值为 $\lambda=0$.下面来求 $T$ 的特征向量.
$$
\begin{pmatrix}
0&0&0&0\\
1&0&0&0\\
0&2&0&0\\
0&0&3&0
\end{pmatrix}
\begin{pmatrix}
  x_1\\
x_2\\
x_3\\
x_4
\end{pmatrix}=
\begin{pmatrix}
  0\\
0\\
0\\
0
\end{pmatrix}.
$$
我们发现,任意非零实数都是特征向量.
\item 不是.因为$T$ 的特征向量张成的空间只有一维,若 $T$ 是可对角化的,$T$
  的特征向量张成的空间应该要有三维.
  \end{itemize}
\end{proof}
\end{document}