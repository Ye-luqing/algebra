\documentclass[a4paper]{article}
\usepackage{amsmath,amsfonts,amsthm,amssymb} \usepackage{bm}
\usepackage{draftwatermark,euler,epigraph}
\setlength{\epigraphwidth}{0.8\textwidth}
\SetWatermarkText{http://blog.sciencenet.cn/u/Yaleking}%设置水印文字
\SetWatermarkLightness{0.8}%设置水印亮度
\SetWatermarkScale{0.35}%设置水印大小
\usepackage{hyperref} \usepackage{geometry} \usepackage{yhmath}
\usepackage{pstricks-add} \usepackage{framed,mdframed}
\usepackage{graphicx,color} \usepackage{mathrsfs,xcolor}
\usepackage[all]{xy} \usepackage{fancybox} \usepackage{xeCJK}
\newtheorem*{theo}{定理} \newtheorem*{exe}{习题} \newenvironment{theorem}
{\bigskip\begin{mdframed}\begin{theo}}
    {\end{theo}\end{mdframed}\bigskip} \newenvironment{exercise}
{\bigskip\begin{mdframed}\begin{exe}}
    {\end{exe}\end{mdframed}\bigskip}
\geometry{left=2.5cm,right=2.5cm,top=2.5cm,bottom=2.5cm}
\setCJKmainfont[BoldFont=SimHei]{SimSun}
\newcommand{\D}{\displaystyle}\newcommand{\ri}{\Rightarrow}
\newcommand{\ds}{\displaystyle} \renewcommand{\ni}{\noindent}
\newcommand{\pa}{\partial} \newcommand{\Om}{\Omega}
\newcommand{\om}{\omega} \newcommand{\sik}{\sum_{i=1}^k}
\newcommand{\vov}{\Vert\omega\Vert} \newcommand{\Umy}{U_{\mu_i,y^i}}
\newcommand{\lamns}{\lambda_n^{^{\scriptstyle\sigma}}}
\newcommand{\chiomn}{\chi_{_{\Omega_n}}}
\newcommand{\ullim}{\underline{\lim}} \newcommand{\bsy}{\boldsymbol}
\newcommand{\mvb}{\mathversion{bold}} \newcommand{\la}{\lambda}
\newcommand{\La}{\Lambda} \newcommand{\va}{\varepsilon}
\newcommand{\be}{\beta} \newcommand{\al}{\alpha}
\newcommand{\dis}{\displaystyle} \newcommand{\R}{{\mathbb R}}
\renewcommand{\today}{\number\year 年 \number\month 月 \number\day 日}
\newcommand{\N}{{\mathbb N}} \newcommand{\cF}{{\mathcal F}}
\newcommand{\gB}{{\mathfrak B}} \newcommand{\eps}{\epsilon}
\renewcommand\refname{参考文献}\renewcommand\figurename{图}
\usepackage[]{caption2} \renewcommand{\captionlabeldelim}{}
\begin{document}
\title{\huge{\bf{线性代数,习题 12}}} \author{\small{叶卢庆\footnote{叶
      卢庆(1992---),E-mail:yeluqingmathematics@gmail.com}}}
\maketitle\ni
\begin{exercise}
令 $P_2(\mathbf{R})$ 为所有次数不超过
$2$ 的实系数多项式形成的线性空间.在 $P_2(\mathbf{R})$ 上定义内积
$$
\langle f,g\rangle:=\int_0^1f(x)g(x)dx.
$$
\begin{itemize}
\item 求出 $P_2(\mathbf{R})$ 的一组标准正交基.
\item 求出 $(1,x)^{\bot}$ 的一个基.
\end{itemize}
\end{exercise}
\begin{proof}[\textbf{证明}]
  \begin{itemize}
  \item $P_2(\mathbf{R})$ 是一个三维线性空间,$(1,x,x^2)=(v_1,v_2,v_3)$
    是一组有序基.显然这组基并非正交基,但是我们可以利用Gram-Schimidt正
    交化过程将这组基正交化.令$w_1=v_1$,
$$
w_2=v_2-\lambda_1v_{1,1}=x-\lambda_{1,1},
$$
且 $w_2$ 与 $v_1$ 正交,因此
$$
\int_0^1x-\lambda dx=0\ri \lambda_{1,1}=\frac{1}{2}.
$$
因此 $w_2=x-\frac{1}{2}$.令 $w_3$ 与 $w_1,w_2$ 都正交,且
$$
w_3=v_3-\lambda_{1,2}v_1-\lambda_{2,2}v_2=x^2-\lambda_{1,2}-\lambda_{2,2}x,
$$
且
$$
\begin{cases}
  \int _0^1x^2-\lambda_{1,2}-\lambda_{2,2}xdx=0,\\
\int_0^1(x-\frac{1}{2})(x^2-\lambda_{1,2}-\lambda_{2,2}x)dx=0.
\end{cases}\ri
\begin{cases}
  \int _0^1x^2-\lambda_{1,2}-\lambda_{2,2}xdx=0,\\
\int_0^1x^3-\lambda_{1,2}x-\lambda_{2,2}x^{2}dx=0.
\end{cases}\ri \lambda_{1,2}=\frac{-1}{6},\lambda_{2,2}=1.
$$
因此 $(1,x-\frac{1}{2},x^2-x+\frac{1}{6})$ 是 $P_2(\mathbf{R})$ 的一组
正交基.进一步,$(1,12x-6,180x^2-180x+30)$ 是 $P_2(\mathbf{R})$ 的一组标
准正交基.
\item $\{x^2\}$.
  \end{itemize}
\end{proof}
\end{document}