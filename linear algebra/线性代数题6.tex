\documentclass[a4paper]{article}
\usepackage{amsmath,amsfonts,amsthm,amssymb}
\usepackage{bm}
\usepackage{draftwatermark}
\SetWatermarkText{http://blog.sciencenet.cn/u/Yaleking}%设置水印文字
\SetWatermarkLightness{0.8}%设置水印亮度
\SetWatermarkScale{0.35}%设置水印大小
\usepackage{hyperref}
\usepackage{geometry}
\usepackage{yhmath}
\usepackage{pstricks-add}
\usepackage{framed,mdframed}
\usepackage{graphicx,color} 
\usepackage{mathrsfs,xcolor} 
\usepackage[all]{xy}
\usepackage{fancybox} 
\usepackage{xeCJK}
\newtheorem*{theo}{Theorem}
\newtheorem*{exe}{Exercise}
\newenvironment{theorem}
{\bigskip\begin{mdframed}\begin{theo}}
    {\end{theo}\end{mdframed}\bigskip}
\newenvironment{exercise}
{\bigskip\begin{mdframed}\begin{exe}}
    {\end{exe}\end{mdframed}\bigskip}
\geometry{left=2.5cm,right=2.5cm,top=2.5cm,bottom=2.5cm}
\setCJKmainfont[BoldFont=SimHei]{SimSun}
\newcommand{\D}{\displaystyle}\newcommand{\ri}{\Rightarrow}
\newcommand{\ds}{\displaystyle} \renewcommand{\ni}{\noindent}
\newcommand{\pa}{\partial} \newcommand{\Om}{\Omega}
\newcommand{\om}{\omega} \newcommand{\sik}{\sum_{i=1}^k}
\newcommand{\vov}{\Vert\omega\Vert} \newcommand{\Umy}{U_{\mu_i,y^i}}
\newcommand{\lamns}{\lambda_n^{^{\scriptstyle\sigma}}}
\newcommand{\chiomn}{\chi_{_{\Omega_n}}}
\newcommand{\ullim}{\underline{\lim}} \newcommand{\bsy}{\boldsymbol}
\newcommand{\mvb}{\mathversion{bold}} \newcommand{\la}{\lambda}
\newcommand{\La}{\Lambda} \newcommand{\va}{\varepsilon}
\newcommand{\be}{\beta} \newcommand{\al}{\alpha}
\newcommand{\dis}{\displaystyle} \newcommand{\R}{{\mathbb R}}
\newcommand{\N}{{\mathbb N}} \newcommand{\cF}{{\mathcal F}}
\newcommand{\gB}{{\mathfrak B}} \newcommand{\eps}{\epsilon}
\renewcommand\refname{参考文献}\renewcommand\figurename{图}
\usepackage[]{caption2} 
\renewcommand{\captionlabeldelim}{}
\begin{document}
\title{\huge{\bf{Linear algebra,Exercise 6}}} \author{\small{叶卢
    庆\footnote{Luqing
      Ye(1992---),E-mail:yeluqingmathematics@gmail.com}}}
\maketitle
\begin{exercise}
  Let $V$ be a three-dimensional vector space with an ordered basis
  $\beta:=(v_1,v_2,v_3)$.Let $\gamma$ be the ordered basis
  $\gamma:=((1,1,0),(1,0,0),(0,0,1))$ of $\mathbf{R}^3$.Let $T:V\to
  \mathbf{R}^3$ be a linear transformation whose matrix representation
  $[T]_{\beta}^{\gamma}$ is given by 
$$
[T]_{\beta}^{\gamma}=\begin{pmatrix}
  0&0&1\\
0&1&0\\
1&0&0
\end{pmatrix}.
$$
Compute $T(v_1+2v_2+3v_3)$.
\end{exercise}
\begin{proof}[\textbf{Solve}]
  \begin{align*}
    T(v_1+2v_2+3v_3)&=T(v_1)+2T(v_2)+3T(v_3)
\\&=(0,0,1)+2(1,0,0)+3(1,1,0)
\\&=(5,3,1).
  \end{align*}
\end{proof}
\end{document}
