\documentclass[a4paper]{article}
\usepackage{amsmath,amsfonts,amsthm,amssymb} \usepackage{bm,euler}
\usepackage{draftwatermark}
\SetWatermarkText{http://blog.sciencenet.cn/u/Yaleking}%设置水印文字
\SetWatermarkLightness{0.8}%设置水印亮度
\SetWatermarkScale{0.35}%设置水印大小
\usepackage{hyperref} \usepackage{geometry} \usepackage{yhmath}
\usepackage{pstricks-add} \usepackage{framed,mdframed}
\usepackage{graphicx,color} \usepackage{mathrsfs,xcolor}
\usepackage[all]{xy} \usepackage{fancybox} \usepackage{xeCJK}
\newtheorem*{theo}{定理} 
\newtheorem*{exa}{习题}
\newtheorem*{rem}{评论}
\newenvironment{theorem}
{\bigskip\begin{mdframed}\begin{theo}}
    {\end{theo}\end{mdframed}\bigskip}
\newenvironment{remark}
{\bigskip\begin{mdframed}\begin{rem}}
    {\end{rem}\end{mdframed}\bigskip} 
\newenvironment{example}
{\bigskip\begin{mdframed}\begin{exa}}
    {\end{exa}\end{mdframed}\bigskip}
\geometry{left=2.5cm,right=2.5cm,top=2.5cm,bottom=2.5cm}
\setCJKmainfont[BoldFont=SimHei]{SimSun}
\newcommand{\D}{\displaystyle}\newcommand{\ri}{\Rightarrow}
\newcommand{\ds}{\displaystyle} \renewcommand{\ni}{\noindent}
\newcommand{\pa}{\partial} \newcommand{\Om}{\Omega}
\newcommand{\om}{\omega} \newcommand{\sik}{\sum_{i=1}^k}
\newcommand{\vov}{\Vert\omega\Vert} \newcommand{\Umy}{U_{\mu_i,y^i}}
\newcommand{\lamns}{\lambda_n^{^{\scriptstyle\sigma}}}
\newcommand{\chiomn}{\chi_{_{\Omega_n}}}
\newcommand{\ullim}{\underline{\lim}} \newcommand{\bsy}{\boldsymbol}
\newcommand{\mvb}{\mathversion{bold}} \newcommand{\la}{\lambda}
\newcommand{\La}{\Lambda} \newcommand{\va}{varepsilon}
\newcommand{\be}{\beta} \newcommand{\al}{\alpha}
\newcommand{\dis}{\displaystyle} \newcommand{\R}{{\mathbb R}}
\renewcommand{\today}{\number\year 年 \number\month 月 \number\day 日}
\newcommand{\N}{{\mathbb N}} \newcommand{\cF}{{\mathcal F}}
\newcommand{\gB}{{\mathfrak B}} \newcommand{\eps}{\epsilon}\newcommand{\op}{\operatorname}
\renewcommand\refname{参考文献}\renewcommand\figurename{图}
\usepackage[]{caption2} \renewcommand{\captionlabeldelim}{}
\begin{document}
\title{\huge{\bf{利用Cauchy的方法化简二次型}}} \author{\small{叶卢
    庆\footnote{叶卢庆(1992---),男,杭州师范大学理学院数学与应用数学专业
      本科在读,E-mail:yeluqingmathematics@gmail.com}}}
\maketitle\ni
这道题目是Victor J.Katz 著的《数学史通论》习题15.40.
\begin{example}
  用Cauchy的方法,找一个正交变换把二次型$2x^2+6xy+5y^2$转换成平方的和或
  者差.
\end{example}
\begin{proof}[\textbf{解}]
  设$f(x,y)=2x^2+6xy+5y^2=k$,其中$k$是一个待定的实数,$f$是二元实变量函
  数.$f(x,y)=k$在平面直角坐标系上的图像是一个椭圆.为了将
  $f(x,y)$化为平方和或者差,我们需要将椭圆摆正成为中学里常见的标准的椭
  圆.为此我们需要确定椭圆$f(x,y)=k$的长轴和短轴.存在以原点为中心的
  圆$g(x,y)=x^2+y^2=r$,该圆与椭圆相切.在切点处,圆的切线和椭圆的切线重合.设切
  点为$(x_{0},y_{0})$,根据隐函数定理,可得
$$
\left.\frac{\pa y}{\pa x}\right|_{(x,y)=(x_0,y_0)}=-\frac{\frac{\pa g}{\pa x}(x_{0},y_{0})}{\frac{\pa g}{\pa y}(x_0,y_0)}=-\frac{\frac{\pa f}{\pa x}(x_{0},y_{0})}{\frac{\pa f}{\pa y}(x_0,y_0)}=\lambda,
$$
也就是说,
$$
x_0^2+x_0y_0=y_0^2.
$$
解得 $\frac{y_0}{x_0}=\frac{1+\sqrt{5}}{2}$或$\frac{1-\sqrt{5}}{2}$,
可见,该椭圆的两条轴所指示的向量分别为
$(1,\frac{1+\sqrt{5}}{2}),(1,\frac{1-\sqrt{5}}{2})$.设
$\alpha=((1,0),(0,1))=(v_1,v_2)$是$\mathbf{R}^2$的一组标准有序
基
,$\beta=((\frac{\sqrt{2}}{\sqrt{5+\sqrt{5}}},\frac{\sqrt{5}+1}{\sqrt{10+2
  \sqrt{5}}}),(\frac{\sqrt{2}}{\sqrt{5-\sqrt{5}}},\frac{1-\sqrt{5}}{\sqrt{10-2
\sqrt{5}}}))=(w_1,w_2)$是
$\mathbf{R}^2$的另一组标准有序基.易得
$$
[I]_{\beta}^{\alpha}=
\begin{pmatrix}
  \frac{\sqrt{2}}{\sqrt{5+\sqrt{5}}}&\frac{\sqrt{2}}{\sqrt{5-\sqrt{5}}}\\
\frac{\sqrt{5}+1}{\sqrt{10+2
  \sqrt{5}}}&\frac{1-\sqrt{5}}{\sqrt{10-2
\sqrt{5}}}
\end{pmatrix},
$$
$$
[I]_{\alpha}^{\beta}=
\begin{pmatrix}
  \frac{\sqrt{15}-\sqrt{3}}{6}&\frac{\sqrt{2}}{2}\\
\frac{\sqrt{15}+\sqrt{3}}{6}&-\frac{\sqrt{2}}{2}
\end{pmatrix}.
$$
则
$$
\begin{pmatrix}
  x\\
y
\end{pmatrix}=\begin{pmatrix}
  \frac{\sqrt{2}}{\sqrt{5+\sqrt{5}}}&\frac{\sqrt{2}}{\sqrt{5-\sqrt{5}}}\\
\frac{\sqrt{5}+1}{\sqrt{10+2
  \sqrt{5}}}&\frac{1-\sqrt{5}}{\sqrt{10-2
\sqrt{5}}}
\end{pmatrix}
\begin{pmatrix}
  x'\\
y'
\end{pmatrix}
$$
代入二次型,可得
$$
\frac{25+11 \sqrt{5}}{5+\sqrt{5}}x'^2+\frac{25-11 \sqrt{5}}{5-\sqrt{5}}y'^2=k,
$$
即
$$
\frac{7+3 \sqrt{5}}{2}x'^2+\frac{7-3 \sqrt{5}}{2}y'^2=k.
$$
\end{proof}
\end{document}





























