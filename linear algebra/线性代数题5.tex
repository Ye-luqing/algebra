\documentclass[a4paper]{article}
\usepackage{amsmath,amsfonts,amsthm,amssymb}
\usepackage{bm}
\usepackage{draftwatermark}
\SetWatermarkText{http://blog.sciencenet.cn/u/Yaleking}%设置水印文字
\SetWatermarkLightness{0.8}%设置水印亮度
\SetWatermarkScale{0.35}%设置水印大小
\usepackage{hyperref}
\usepackage{geometry}
\usepackage{yhmath}
\usepackage{pstricks-add}
\usepackage{framed,mdframed}
\usepackage{graphicx,color} 
\usepackage{mathrsfs,xcolor} 
\usepackage[all]{xy}
\usepackage{fancybox} 
\usepackage{xeCJK}
\newtheorem*{theo}{Theorem}
\newtheorem*{exe}{Exercise}
\newenvironment{theorem}
{\bigskip\begin{mdframed}\begin{theo}}
    {\end{theo}\end{mdframed}\bigskip}
\newenvironment{exercise}
{\bigskip\begin{mdframed}\begin{exe}}
    {\end{exe}\end{mdframed}\bigskip}
\geometry{left=2.5cm,right=2.5cm,top=2.5cm,bottom=2.5cm}
\setCJKmainfont[BoldFont=SimHei]{SimSun}
\newcommand{\D}{\displaystyle}\newcommand{\ri}{\Rightarrow}
\newcommand{\ds}{\displaystyle} \renewcommand{\ni}{\noindent}
\newcommand{\pa}{\partial} \newcommand{\Om}{\Omega}
\newcommand{\om}{\omega} \newcommand{\sik}{\sum_{i=1}^k}
\newcommand{\vov}{\Vert\omega\Vert} \newcommand{\Umy}{U_{\mu_i,y^i}}
\newcommand{\lamns}{\lambda_n^{^{\scriptstyle\sigma}}}
\newcommand{\chiomn}{\chi_{_{\Omega_n}}}
\newcommand{\ullim}{\underline{\lim}} \newcommand{\bsy}{\boldsymbol}
\newcommand{\mvb}{\mathversion{bold}} \newcommand{\la}{\lambda}
\newcommand{\La}{\Lambda} \newcommand{\va}{\varepsilon}
\newcommand{\be}{\beta} \newcommand{\al}{\alpha}
\newcommand{\dis}{\displaystyle} \newcommand{\R}{{\mathbb R}}
\newcommand{\N}{{\mathbb N}} \newcommand{\cF}{{\mathcal F}}
\newcommand{\gB}{{\mathfrak B}} \newcommand{\eps}{\epsilon}
\renewcommand\refname{参考文献}\renewcommand\figurename{图}
\usepackage[]{caption2} 
\renewcommand{\captionlabeldelim}{}
\begin{document}
\title{\huge{\bf{Linear algebra,Exercise 5}}} \author{\small{叶卢
    庆\footnote{Luqing
      Ye(1992---),E-mail:yeluqingmathematics@gmail.com}}}
\maketitle
\begin{exercise}\footnote{From
    http://www.math.ucla.edu/~tao/resource/general/115a.3.02f/midterm.pdf}
Let $T:\mathbf{R}^4\to \mathbf{R}^4$ be the transformation
$$
T(x_1,x_2,x_3,x_4)=(0,x_1,x_2,x_3).
$$
\begin{itemize}
\item What is the rank and nullity of $T$?
\item Let $\beta=((1,0,0,0),(0,1,0,0),(0,0,1,0),(0,0,0,1))$ be the
  standard ordered basis for $T$.Compute
  $[T]_{\beta}^{\beta}$,$[T^2]_{\beta}^{\beta}$,$[T^3]_{\beta}^{\beta}$,and
  $[T^4]_{\beta}^{\beta}$.
\end{itemize}
\end{exercise}
\begin{proof}[\textbf{Solve}]
  \begin{itemize}
  \item Let $v_1=(1,0,0,0),v_2=(0,1,0,0),v_3=(0,0,1,0),v_4=(0,0,0,1)$.Then
$$
T(v_1)=(0,1,0,0)=v_2,T(v_2)=(0,0,1,0)=v_3,T(v_3)=(0,0,0,1)=v_4,T(v_4)=(0,0,0,0)=\mathbf{0}.
$$
So the rank of $T$ is 3,the nullity of $T$ is 1.
\item
$$
[T]_{\beta}^{\beta}=\begin{pmatrix}
  0&0&0&0\\
1&0&0&0\\
0&1&0&0\\
0&0&1&0
\end{pmatrix}.
$$
$$
T^2(v_1)=T(v_2)=v_3,T^2(v_2)=T(v_3)=v_4,T^2(v_3)=T(v_4)=\mathbf{0},T^2(v_4)=\mathbf{0}.
$$
So
$$
[T^2]_{\beta}^{\beta}=\begin{pmatrix}
  0&0&0&0\\
0&0&0&0\\
1&0&0&0\\
0&1&0&0
\end{pmatrix}.
$$
$$
T^3(v_1)=T(v_3)=v_4,T^3(v_2)=T(v_4)=\mathbf{0},T^3(v_3)=T^3(v_4)=\mathbf{0}.
$$
So
$$
[T^3]_{\beta}^{\beta}=\begin{pmatrix}
  0&0&0&0\\
0&0&0&0\\
0&0&0&0\\
1&0&0&0
\end{pmatrix}.
$$
And
$$
[T^4]_{\beta}^{\beta}=\begin{pmatrix}
  0&0&0&0\\
0&0&0&0\\
0&0&0&0\\
0&0&0&0
\end{pmatrix}.
$$
\end{itemize}
\end{proof}
\end{document}
