\documentclass{article}
\usepackage{amsmath}
\usepackage{amsthm}
\usepackage{amssymb}
\usepackage{amsfonts}
\usepackage{graphicx,color}
\usepackage{mathrsfs}
\usepackage[all]{xy}
\usepackage{fancybox}
\usepackage{CJKutf8}
\newtheorem{thm}{定理}
\newtheorem{lem}{引理}
\newtheorem{note}{注}
\newtheorem{exa}{例}
\newtheorem{exe}{习题}
\begin{document}
\renewcommand{\refname}{参考文献}
\begin{CJK}{UTF8}{gbsn}
  \title{陶哲轩实分析习题17.1.4}\author{叶卢庆}\date{}
  \maketitle\noindent
  设$T:\mathbf{R}^n\to\mathbf{R}^m$是线性变换,证明,存在数$M>0$,使得
  对于一切$x\in\mathbf{R}^n$,$||T(x)||\leq M||x||$.\\\\
\begin{proof}[证明]
  设$T$对应的矩阵为
  \begin{equation}
    \begin{pmatrix}
      a_{11}&\cdots&a_{1n}\\
\vdots&\cdots&\vdots\\
a_{m1}&\cdots&a_{mn}\\
    \end{pmatrix}
  \end{equation}
$\forall x=(a_1,\cdots,a_n)\in\mathbf{R}^n$,
\begin{equation}
  T(x)=\begin{pmatrix}
    a_{11}&\cdots&a_{1n}\\
\vdots&\cdots&\vdots\\
a_{m1}&\cdots&a_{mn}\\
  \end{pmatrix}\begin{pmatrix}
    a_1\\
\vdots\\
a_n\\
  \end{pmatrix}
\end{equation}
则
\begin{equation}
||T(x)|| ^2= \sum_{k=1}^m(a_{k1}a_1+\cdots+a_{kn}a_n)^2
\end{equation}
根据柯西不等式,
\begin{equation}
(a_{k1}a_1+\cdots+a_{kn}a_n)^2\leq (a_{k1}^2+\cdots+a_{kn}^2)(a_1^2+\cdots+a_n^2)
\end{equation}
因此
\begin{equation}
  \sum_{k=1}^m(a_{k1}a_1+\cdots+a_{kn}a_n)^2\leq (a_1^2+\cdots+a_n^2)(\sum_{i=1}^m\sum_{j=1}^na_{ij}^2)
\end{equation}
可见,令
\begin{equation}
M=\sqrt{\sum_{i=1}^m\sum_{j=1}^na_{ij}^2}  
\end{equation}
即可.





  
\end{proof}

  
  
  
  
  \bibliographystyle{unsrt}
  % \bibliography{}
\end{CJK}
\end{document}
