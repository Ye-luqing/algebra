\documentclass[a4paper]{article}
\usepackage{amsmath,amsfonts,amsthm,amssymb} \usepackage{bm}
\usepackage{draftwatermark}
\SetWatermarkText{http://blog.sciencenet.cn/u/Yaleking}%设置水印文字
\SetWatermarkLightness{0.8}%设置水印亮度
\SetWatermarkScale{0.35}%设置水印大小
\usepackage{hyperref} \usepackage{geometry} \usepackage{yhmath,euler}
\usepackage{pstricks-add} \usepackage{framed,mdframed}
\usepackage{graphicx,color} \usepackage{mathrsfs,xcolor}
\usepackage[all]{xy} \usepackage{fancybox} \usepackage{xeCJK}
\newtheorem*{theo}{定理} 
\newtheorem{exa}{例}
\newenvironment{theorem}
{\bigskip\begin{mdframed}\begin{theo}}
    {\end{theo}\end{mdframed}\bigskip} 
\newenvironment{example}
{\bigskip\begin{mdframed}\begin{exa}}
    {\end{exa}\end{mdframed}\bigskip}
\geometry{left=2.5cm,right=2.5cm,top=2.5cm,bottom=2.5cm}
\setlength\parindent{0pt}
\setCJKmainfont[BoldFont=SimHei]{SimSun}
\newcommand{\D}{\displaystyle}\newcommand{\ri}{\Rightarrow}
\newcommand{\ds}{\displaystyle} \renewcommand{\ni}{\noindent}
\newcommand{\pa}{\partial} \newcommand{\Om}{\Omega}
\newcommand{\om}{\omega} \newcommand{\sik}{\sum_{i=1}^k}
\newcommand{\vov}{\Vert\omega\Vert} \newcommand{\Umy}{U_{\mu_i,y^i}}
\newcommand{\lamns}{\lambda_n^{^{\scriptstyle\sigma}}}
\newcommand{\chiomn}{\chi_{_{\Omega_n}}}
\newcommand{\ullim}{\underline{\lim}} \newcommand{\bsy}{\boldsymbol}
\newcommand{\mvb}{\mathversion{bold}} \newcommand{\la}{\lambda}
\newcommand{\La}{\Lambda} \newcommand{\va}{\varepsilon}
\newcommand{\be}{\beta} \newcommand{\al}{\alpha}
\newcommand{\dis}{\displaystyle} \newcommand{\R}{{\mathbb R}}
\renewcommand{\today}{\number\year 年 \number\month 月 \number\day 日}
\newcommand{\N}{{\mathbb N}} \newcommand{\cF}{{\mathcal F}}
\newcommand{\gB}{{\mathfrak B}} \newcommand{\eps}{\epsilon}
\renewcommand\refname{参考文献}\renewcommand\figurename{图}
\usepackage[]{caption2} \renewcommand{\captionlabeldelim}{}
\begin{document}
\title{\huge{\bf{逐步转轴法化实二次型为标准型}}} \author{\small{叶卢
    庆\footnote{叶卢庆(1992---),男,杭州师范大学理学院数学与应用数学专业
      本科在读,E-mail:yeluqingmathematics@gmail.com}}}
\maketitle
我们先来看两个例子.通过这两个例子,我们来演示如何使用逐步转轴法消去二次
型中的各交叉项,最后成为标准型.
\begin{example}[中国科学技术大学2012年硕士学位研究生入学考试试题]
  在$\mathbf{R}^3$中,方程$xy-yz+zx=1$所表示的二次曲面类型为?
\end{example}
\begin{proof}[\textbf{解}]
  我们先考虑化去交叉项$xy$.令
$$
\begin{pmatrix}
  x\\
  y\\
  z
\end{pmatrix}=
\begin{pmatrix}
  \cos\alpha&-\sin\alpha&0\\
  \sin\alpha&\cos\alpha&0\\
  0&0&1
\end{pmatrix}
\begin{pmatrix}
  x'\\
  y'\\
  z'
\end{pmatrix}
$$
代入二次型
\begin{equation}
  \label{eq:1}
  xy-yz+zx
\end{equation}
整理后可得
\begin{equation}
  \label{eq:2}
  x'^{2}\cos\alpha\sin\alpha-y'^{2}\sin\alpha\cos\alpha+x'y'\cos
  2\alpha+y'z'(\cos\alpha-\sin\alpha)+x'z'(\cos\alpha-\sin\alpha).
\end{equation}
令$x'y'$前面的系数等于$0$,即让$\cos 2\alpha=0$,此时$\alpha$可以
为$\frac{\pi}{4}$.在这个时候,$y'z'$和$x'z'$前面的系数也恰好
为$0$.此时,二次型\eqref{eq:2}可以化为
$$
\frac{1}{2}x'^2-\frac{1}{2}y'^2.
$$
也即,通过正交替换,方程$xy-yz+zx=1$变成了$x'^2-y'^2=2$.这是一个双曲柱面.
\end{proof}
\begin{example}
  将二次型
$$
f(x_{1},x_2,x_3)=2x_{1}^{2} + 5x_{2}^{2} +
5x_{3}^{2}+4x_{1}x_{2}-4x_{1}x_{3} - 8x_{2}x_3
$$
化为标准型.
\end{example}
\begin{proof}[\textbf{解}]
  我们先考虑化去交叉项$x_1x_2$.令
$$
\begin{pmatrix}
  x_1\\
  x_2\\
  x_3
\end{pmatrix}=
\begin{pmatrix}
  \cos\alpha&-\sin\alpha&0\\
  \sin\alpha&\cos\alpha&0\\
  0&0&1
\end{pmatrix}
\begin{pmatrix}
  x_1'\\
  x_2'\\
  x_3'
\end{pmatrix},
$$
代入二次型$f(x_1,x_2,x_3)$,整理可得
\begin{align*}
  f(x_1,x_2,x_3)=&
  (2+3\sin^2\alpha+4\sin\alpha\cos\alpha)x_1'^2+(4\cos 2\alpha+3\sin
  2\alpha)x_1'x_2'+(2+3\cos^2\alpha-4\sin\alpha\cos\alpha)x_2'^2\\&-(4\cos\alpha+8\sin\alpha)x_1'x_3'+(4\sin\alpha-8\cos\alpha)x_2'x_3'+5x_3'^2.
\end{align*}
令$x_1'x_2'$前面的系数$4\cos 2\alpha+3\sin
2\alpha$为$0$,解得$\sin\alpha=\frac{2
  \sqrt{5}}{5}$,$\cos\alpha=\frac{\sqrt{5}}{5}$是一组解.这个时候,二次
型$f(x_1,x_2,x_3)$成为
\begin{align*}
  6x_1'^2+x_2'^2-4 \sqrt{5}x_1'x_3'+5x_3'^2.
\end{align*}
我们很幸运地发现,$x_2'x_3'$同时也消失了.然后我们试图消去最后的交叉
项$x_1'x_3'$.令
$$
\begin{pmatrix}
  x_1'\\
  x_2'\\
  x_3'
\end{pmatrix}=
\begin{pmatrix}
  \cos\beta&0&-\sin\beta\\
  0&1&0\\
  \sin\beta&0&\cos\beta
\end{pmatrix}
\begin{pmatrix}
  x_1''\\
  x_2''\\
  x_3''
\end{pmatrix},
$$
则二次型$f(x_1,x_2,x_3)$成为
$$
(5+\cos^2\beta-4 \sqrt{5}\cos\beta\sin\beta)x_1''^2+x_2''^2+(-\sin
2\beta-4 \sqrt{5}\cos 2\beta)x_1''x_3''+(5+\sin^2\beta+4
\sqrt{5}\sin\beta\cos\beta)x_3''^2.
$$
令 $x_1''x_3''$前面的系数为零,即让$-\sin 2\beta-4 \sqrt{5}\cos
2\beta=0$,解得$\cos\beta=\frac{2}{3}$,$\sin\beta=\frac{\sqrt{5}}{3}$是
一组解.此时,二次型$f(x_1,x_2,x_3)$成了
$$
x_1''^2+x_2''^2+10x_3''^2.
$$
这样就成功地把二次型$f(x_1,x_2,x_3)$化为了标准型.其中
\begin{align*}
  \begin{pmatrix}
    x_1''\\
    x_2''\\
    x_3''
  \end{pmatrix}=\begin{pmatrix}
    \cos-\beta&0&-\sin-\beta\\
    0&1&0\\
    \sin-\beta&0&\cos-\beta
  \end{pmatrix}
  \begin{pmatrix}
    x_1'\\
    x_2'\\
    x_3'
  \end{pmatrix}&=\begin{pmatrix}
    \cos-\beta&0&-\sin-\beta\\
    0&1&0\\
    \sin-\beta&0&\cos-\beta
  \end{pmatrix}\begin{pmatrix}
    \cos-\alpha&-\sin-\alpha&0\\
    \sin-\alpha&\cos-\alpha&0\\
    0&0&1
  \end{pmatrix}
  \begin{pmatrix}
    x\\
    y\\
    z
  \end{pmatrix}\\&=\begin{pmatrix}
    \cos\beta&0&\sin\beta\\
    0&1&0\\
    -\sin\beta&0&\cos\beta
  \end{pmatrix}\begin{pmatrix}
    \cos\alpha&\sin\alpha&0\\
    -\sin\alpha&\cos\alpha&0\\
    0&0&1
  \end{pmatrix}
  \begin{pmatrix}
    x\\
    y\\
    z
  \end{pmatrix}\\&=
  \begin{pmatrix}
    \frac{2 \sqrt{5}}{15}&\frac{4 \sqrt{5}}{15}&\frac{\sqrt{5}}{3}\\
    -\frac{2 \sqrt{5}}{5}&\frac{\sqrt{5}}{5}&0\\
    -\frac{1}{3}&-\frac{2}{3}&\frac{2}{3}
  \end{pmatrix}
  \begin{pmatrix}
    x\\
    y\\
    z
  \end{pmatrix}.
\end{align*}
\end{proof}
对于一般的实二次型
$$
f(x_1,x_2,\cdots,x_n)=a_{11}x_1^2+a_{22}x_2^2+\cdots+a_{nn}x_n^2+2a_{12}x_1x_2+2a_{13}x_1x_3+\cdots+2a_{n-1,n}x_{n-1}x_n
$$
来说,我们可以先考虑通过转轴去掉交叉项$x_1x_2$,为此,令
$$
\begin{cases}
  x_1=x_{1}^{(1,2)}\cos\alpha_{1,2}-x_{2}^{(1,2)}\sin\alpha_{1,2},\\
  x_2=x_{1}^{(1,2)}\sin\alpha_{1,2}+x_{2}^{(1,2)}\cos\alpha_{1,2},\\
  x_3=x_3^{(1,2)},\\
  \vdots\\
  x_n=x_n^{(1,2)}.
\end{cases}
$$
这样,经过替换后,可得
$$
f(x_1,x_2,\cdots,x_n)=f_{1,2}(x_1^{(1,2)},x_2^{(1,2)},\cdots,x_n^{(1,2)}),
$$
通过选取特定的$\alpha_{1,2}$,必定能
使$f_{1,2}(x_1^{(1,2)},x_2^{(1,2)},\cdots,x_n^{(1,2)})$中不
含$x_1^{(1,2)}x_2^{(1,2)}$这一项.此
时,$f_{1,2}(x_1^{(1,2)},x_2^{(1,2)},\cdots,x_n^{(1,2)})$中交叉项只剩
下
$x_1^{(1,2)}x_3^{(1,2)},\cdots,x_1^{(1,2)}x_n^{(1,2)},x_2^{(1,2)}x_{3}^{(1,2)},\cdots,x_2^{(1,2)}x_{n}^{(1,2)},\cdots,x_{n-1}^{(1,2)}x_n^{(1,2)}$.然
后我们再考虑通过转轴去掉交叉项$x_1^{(1,2)}x_3^{(1,2)}$,为此,只需要令
$$
\begin{cases} 
  x_1^{(1,2)}=x_1^{(1,3)}\cos\alpha_{1,3}-x_3^{(1,3)}\sin\alpha_{1,3},\\
  x_2^{(1,2)}=x_2^{(1,3)},\\
  x_3^{(1,2)}=x_1^{(1,3)}\sin\alpha_{1,3}+x_3^{(1,3)}\cos\alpha_{1,3},\\
  \vdots\\
  x_n^{(1,2)}=x_n^{(1,3)}
\end{cases},
$$
经过替换后,
$$
f_{1,2}(x_1^{(1,2)},x_2^{(1,2)},\cdots,x_n^{(1,2)})=f_{1,3}(x_1^{(1,3)},x_2^{(1,3)},\cdots,x_n^{(1,3)}),
$$
通过选取特定的$\alpha_{1,3}$,必定能使
得$f_{1,3}(x_1^{(1,3)},x_2^{(1,3},\cdots,x_n^{(1,3})$中的交叉
项$x_1^{(1,3}x_3^{(1,3)}$消去,此时,
$f_{1,3}(x_1^{(1,3)},x_2^{(1,3)},\cdots,x_n^{(1,3)})$中交叉项只剩
下
$x_1^{(1,3)}x_4^{(1,3)},\cdots,x_1^{(1,3)}x_n^{(1,3)},x_2^{(1,3)}x_{3}^{(1,3)},\cdots,x_2^{(1,3)}x_{n}^{(1,3)},\cdots,x_{n-1}^{(1,3)}x_n^{(1,3)}$.之
后经过若干次类似的操作,会消去所有的交叉项,剩下的二次型
$$
f_{n-1,n}(x_1^{(n-1,n)},x_2^{(n-1,n)},\cdots,x_n^{(n-1,n)}),
$$
只含有平方项,而不含有交叉项.而且,
\begin{align*}
  \begin{pmatrix}
    x_1^{(n-1,n)}\\
    x_2^{(n-1,n)}\\
    \vdots\\
    x_n^{(n-1,n)}
  \end{pmatrix}= A_{n-1,n}\cdots A_{13}A_{12}
  \begin{pmatrix}
    x_1\\
    x_2\\
    \vdots\\
    x_n
  \end{pmatrix},
\end{align*}
其中矩阵$A_{ij}$表示将直角坐标平面$x_i^{(i,j-1)}Ox_j^{(i,j-1)}$逆时针旋
转$\alpha_{i,j-1}$角度(当$i< j-1$时)或者将直角坐标平
面$x_i^{(i-1,n)}Oy_j^{(i-1,n)}$旋转$\alpha_{i-1,n}$角度(当$i=j-1$)的同
时,保持其余的坐标轴不动的线性变换.这是高维空间中的旋转,又叫
做\href{http://en.wikipedia.org/wiki/Givens_rotation}{Givens旋转}.\\\\

我们发现,$A_{n-1,n}\cdots A_{13}A_{12}$表示一个正交矩阵,该正交矩阵代表
若干个Givens旋转的复合.可见,使用逐步转轴法化二次型为标准型的总体效果,
就是用正交变换化二次型为标准型.
\end{document}
