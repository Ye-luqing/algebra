\documentclass[a4paper]{article}
\usepackage{amsmath,amsfonts,amsthm,amssymb}
\usepackage{bm,euler}
\usepackage{hyperref}
\usepackage{geometry}
\usepackage{yhmath}
\usepackage{pstricks-add}
\usepackage{framed,mdframed}
\usepackage{graphicx,color} 
\usepackage{mathrsfs,xcolor} 
\usepackage[all]{xy}
\usepackage{fancybox} 
\usepackage{xeCJK}
\newtheorem*{theo}{定理}
\newtheorem*{exe}{题目}
\newtheorem*{rem}{评论}
\newtheorem*{lemma}{引理}
\newtheorem*{coro}{推论}
\newtheorem*{exa}{例}
\newenvironment{corollary}
{\bigskip\begin{mdframed}\begin{coro}}
    {\end{coro}\end{mdframed}\bigskip}
\newenvironment{theorem}
{\bigskip\begin{mdframed}\begin{theo}}
    {\end{theo}\end{mdframed}\bigskip}
\newenvironment{exercise}
{\bigskip\begin{mdframed}\begin{exe}}
    {\end{exe}\end{mdframed}\bigskip}
\newenvironment{example}
{\bigskip\begin{mdframed}\begin{exa}}
    {\end{exa}\end{mdframed}\bigskip}
\newenvironment{remark}
{\bigskip\begin{mdframed}\begin{rem}}
    {\end{rem}\end{mdframed}\bigskip}
\geometry{left=2.5cm,right=2.5cm,top=2.5cm,bottom=2.5cm}
\setCJKmainfont[BoldFont=SimHei]{SimSun}
\renewcommand{\today}{\number\year 年 \number\month 月 \number\day 日}
\newcommand{\D}{\displaystyle}\newcommand{\ri}{\Rightarrow}
\newcommand{\ds}{\displaystyle} \renewcommand{\ni}{\noindent}
\newcommand{\ov}{\overrightarrow}
\newcommand{\pa}{\partial} \newcommand{\Om}{\Omega}
\newcommand{\om}{\omega} \newcommand{\sik}{\sum_{i=1}^k}
\newcommand{\vov}{\Vert\omega\Vert} \newcommand{\Umy}{U_{\mu_i,y^i}}
\newcommand{\lamns}{\lambda_n^{^{\scriptstyle\sigma}}}
\newcommand{\chiomn}{\chi_{_{\Omega_n}}}
\newcommand{\ullim}{\underline{\lim}} \newcommand{\bsy}{\boldsymbol}
\newcommand{\mvb}{\mathversion{bold}} \newcommand{\la}{\lambda}
\newcommand{\La}{\Lambda} \newcommand{\va}{\varepsilon}
\newcommand{\be}{\beta} \newcommand{\al}{\alpha}
\newcommand{\dis}{\displaystyle} \newcommand{\R}{{\mathbb R}}
\newcommand{\N}{{\mathbb N}} \newcommand{\cF}{{\mathcal F}}
\newcommand{\gB}{{\mathfrak B}} \newcommand{\eps}{\epsilon}
\renewcommand\refname{参考文献}\renewcommand\figurename{图}
\usepackage[]{caption2} 
\renewcommand{\captionlabeldelim}{}
\setlength\parindent{0pt}
\begin{document}
\title{\huge{\bf{二阶行列式的线性性的几何意义}}} \author{\small{叶卢庆
    \footnote{叶卢庆(1992---),男,杭州师范大学理学院数学与应用数学专业本科在读,E-mail:yeluqingmathematics@gmail.com}}}
\maketitle
所谓二阶行列式的线性性,指的是如下公式:
\begin{equation}
  \label{eq:1}
  \begin{vmatrix}
    a_{11}&a_{12}+b_{12}\\
a_{21}&a_{22}+b_{22}
  \end{vmatrix}=
  \begin{vmatrix}
    a_{11}&a_{12}\\
a_{21}&a_{22}
  \end{vmatrix}+
  \begin{vmatrix}
    a_{11}&b_{12}\\
a_{21}&b_{22}
  \end{vmatrix}.
\end{equation}
其几何意义如图\eqref{fig:1}.为叙述简便起见,我们只考虑一种情形.当向量$\ov{AD}=(a_{12},a_{22}),\ov{AE}=(b_{12},b_{22})$都在向量$\ov{AB}=(a_{11},a_{21})$的
顺时针方向时,
$$
S_{ABFE}+S_{ABCD}=S_{DCHG}+S_{ABCD}=S_{ABHG}-S_{ADG}+S_{BCH}=S_{ABHG}.
$$
这就是式\eqref{eq:1}的几何意义.当然,更好的方式不是使用面积来做,而是使
用外积来做.因为用外积来做不需要分类讨论,而用面积来做需要分类讨论,会显
得比较麻烦(正如在此我们用面积做的时候为了避免叙述麻烦只考虑了一种可能的情
形,即$\ov{AD}$和$\ov{AE}$都在$\ov{AB}$的顺时针方向).\\

当线段$CD$和$AB,HG$不在同一个平面时,平行四边形$ABCD$,$DCHG$,$ABHG$围成
一个没有底面$BCH$和$ADG$的三棱柱.由于$\ov{AB}\times \ov{AD}$和
$ABCD$,$\ov{AB}\times \ov{AE}$和$ABFE$以及$\ov{AB}\times\ov{AG}$和$ABHG$之间的
对偶关系,因此和三个平行四边形能围成三棱柱相对应的,三个向量$\ov{AB}\times\ov{AD}$,$\ov{AB}\times\ov{AE}$和
$\ov{AB}\times\ov{AG}$也能形成一个三角形,如图\eqref{fig:2}所示.其中图
\eqref{fig:2}中三角形的三个内角和图\eqref{fig:1}中三个平行四边形之间围
成的二面角对应相等.特别地,当线段$CD$和线段$AB,HG$共面时,三个向量形成的
三角形会退化成为一个线段,结合
$$
\ov{AB}\times \ov{AG}=\ov{AB}\times \ov{AD}+\ov{AB}\times\ov{AE},
$$
便可得到式\eqref{eq:1}.
\begin{figure}[h]
\newrgbcolor{qqccqq}{0. 0.8 0.}
\newrgbcolor{wwqqcc}{0.4 0. 0.8}
\newrgbcolor{zzttqq}{0.6 0.2 0.}
\newrgbcolor{qqzzff}{0. 0.6 1.}
\psset{xunit=1.0cm,yunit=1.0cm,algebraic=true,dimen=middle,dotstyle=o,dotsize=3pt 0,linewidth=0.8pt,arrowsize=3pt 2,arrowinset=0.25}
\begin{pspicture*}(-2.3,-5.82)(12,2.3)
\pspolygon[linecolor=qqccqq,fillcolor=qqccqq,fillstyle=solid,opacity=0.1](1.,-2.)(2.,1.)(7.,2.)(6.,-1.)
\pspolygon[linecolor=wwqqcc,fillcolor=wwqqcc,fillstyle=solid,opacity=0.1](1.,-2.)(3.,-3.)(4.,0.)(2.,1.)
\pspolygon[linecolor=zzttqq,fillcolor=zzttqq,fillstyle=solid,opacity=0.1](6.,-1.)(8.,-2.)(3.,-3.)(1.,-2.)
\pspolygon[linecolor=qqzzff,fillcolor=qqzzff,fillstyle=solid,opacity=0.1](2.,1.)(9.,1.)(8.,-2.)(1.,-2.)
\psline[linecolor=qqccqq](1.,-2.)(2.,1.)
\psline[linecolor=qqccqq](2.,1.)(7.,2.)
\psline[linecolor=qqccqq](7.,2.)(6.,-1.)
\psline[linecolor=qqccqq](6.,-1.)(1.,-2.)
\psline[linecolor=wwqqcc](1.,-2.)(3.,-3.)
\psline[linecolor=wwqqcc](3.,-3.)(4.,0.)
\psline[linecolor=wwqqcc](4.,0.)(2.,1.)
\psline[linecolor=wwqqcc](2.,1.)(1.,-2.)
\psline[linecolor=zzttqq](6.,-1.)(8.,-2.)
\psline[linecolor=zzttqq](8.,-2.)(3.,-3.)
\psline[linecolor=zzttqq](3.,-3.)(1.,-2.)
\psline[linecolor=zzttqq](1.,-2.)(6.,-1.)
\psline[linecolor=qqzzff](2.,1.)(9.,1.)
\psline[linecolor=qqzzff](9.,1.)(8.,-2.)
\psline[linecolor=qqzzff](8.,-2.)(1.,-2.)
\psline[linecolor=qqzzff](1.,-2.)(2.,1.)
\psline{->}(1.,-2.)(2.,1.)
\psline{->}(1.,-2.)(3.,-3.)
\psline{->}(1.,-2.)(6.,-1.)
\psline{->}(1.,-2.)(8.,-2.)
\psline(7.,2.)(9.,1.)
\rput[tl](0,0.94){$(a_{11},a_{21})$}
\rput[tl](2.76,-3.08){$(b_{12},b_{22})$}
\rput[tl](4.3,-0.68){$(a_{12},a_{22})$}
\rput[tl](7.8,-2.16){$(a_{12}+b_{12},a_{22}+b_{22})$}
\begin{scriptsize}
\psdots[dotsize=1pt 0,dotstyle=*,linecolor=blue](1.,-2.)
\rput[bl](0.68,-2.22){\blue{$A$}}
\psdots[dotsize=1pt 0,dotstyle=*,linecolor=blue](2.,1.)
\rput[bl](2.08,1.04){\blue{$B$}}
\psdots[dotsize=1pt 0,dotstyle=*,linecolor=blue](7.,2.)
\rput[bl](7.08,2.04){\blue{$C$}}
\psdots[dotsize=1pt 0,dotstyle=*,linecolor=blue](6.,-1.)
\rput[bl](6.08,-0.96){\blue{$D$}}
\psdots[dotsize=1pt 0,dotstyle=*,linecolor=blue](3.,-3.)
\rput[bl](3.32,-2.78){\blue{$E$}}
\psdots[dotsize=1pt 0,dotstyle=*,linecolor=blue](4.,0.)
\rput[bl](4.08,0.04){\blue{$F$}}
\psdots[dotsize=1pt 0,dotstyle=*,linecolor=blue](8.,-2.)
\rput[bl](8.08,-1.96){\blue{$G$}}
\psdots[dotsize=1pt 0,dotstyle=*,linecolor=blue](9.,1.)
\rput[bl](9.08,1.04){\blue{$H$}}
\end{scriptsize}
\end{pspicture*}
\caption{}
  \label{fig:1}\label{fig:1}
\end{figure}
\begin{figure}[h]
\psset{xunit=1.0cm,yunit=1.0cm,algebraic=true,dimen=middle,dotstyle=o,dotsize=3pt 0,linewidth=0.8pt,arrowsize=3pt 2,arrowinset=0.25}
\begin{pspicture*}(-1.3,-5.82)(11,4.5)
\psline{->}(8.86,-2.7)(9.14,3.4)
\psline{->}(8.86,-2.7)(6.22,-0.74)
\psline{->}(6.22,-0.74)(9.14,3.4)
\rput[tl](9.18,0.76){$\overrightarrow{AB}\times\overrightarrow{AG}$}
\rput[tl](5.2,-1.58){$\overrightarrow{AB}\times\overrightarrow{AD}$}
\rput[tl](6.24,1.9){$\overrightarrow{AB}\times\overrightarrow{AE}$}
\begin{scriptsize}
\psdots[dotsize=1pt 0,dotstyle=*,linecolor=blue](8.86,-2.7)
\psdots[dotsize=1pt 0,dotstyle=*,linecolor=blue](9.14,3.4)
\psdots[dotsize=1pt 0,dotstyle=*,linecolor=blue](6.22,-0.74)
\end{scriptsize}
\end{pspicture*}
  \caption{}
  \label{fig:2}
\end{figure}
\end{document}


\section{三阶行列式的线性性的几何意义}
三阶行列式的线性性,指的是如下公式:
\begin{equation}
  \label{eq:2}
  \begin{vmatrix}
    a_{11}&a_{12}&a_{13}+b_{13}\\
a_{21}&a_{22}&a_{23}+b_{23}\\
a_{31}&a_{32}&a_{33}+b_{33}
  \end{vmatrix}=
  \begin{vmatrix}
    a_{11}&a_{12}&a_{13}\\
a_{21}&a_{22}&a_{23}\\
a_{31}&a_{32}&a_{33}
  \end{vmatrix}+
  \begin{vmatrix}
    a_{11}&a_{12}&b_{13}\\
a_{21}&a_{22}&b_{23}\\
a_{31}&a_{32}&b_{33}
  \end{vmatrix}.
\end{equation}
