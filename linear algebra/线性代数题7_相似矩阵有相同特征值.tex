\documentclass[a4paper]{article}
\usepackage{amsmath,amsfonts,amsthm,amssymb} \usepackage{bm}
\usepackage{draftwatermark,euler}
\SetWatermarkText{http://blog.sciencenet.cn/u/Yaleking}%设置水印文字
\SetWatermarkLightness{0.8}%设置水印亮度
\SetWatermarkScale{0.35}%设置水印大小
\usepackage{hyperref} \usepackage{geometry} \usepackage{yhmath}
\usepackage{pstricks-add} \usepackage{framed,mdframed}
\usepackage{graphicx,color} \usepackage{mathrsfs,xcolor}
\usepackage[all]{xy} \usepackage{fancybox} \usepackage{xeCJK}
\newtheorem*{theo}{定理} 
\newtheorem*{exe}{Exercise}
\newenvironment{theorem}
{\bigskip\begin{mdframed}\begin{theo}}
    {\end{theo}\end{mdframed}\bigskip} 
\newenvironment{exercise}
{\bigskip\begin{mdframed}\begin{exe}}
    {\end{exe}\end{mdframed}\bigskip}
\geometry{left=2.5cm,right=2.5cm,top=2.5cm,bottom=2.5cm}
\setCJKmainfont[BoldFont=SimHei]{SimSun}
\newcommand{\D}{\displaystyle}\newcommand{\ri}{\Rightarrow}
\newcommand{\ds}{\displaystyle} \renewcommand{\ni}{\noindent}
\newcommand{\pa}{\partial} \newcommand{\Om}{\Omega}
\newcommand{\om}{\omega} \newcommand{\sik}{\sum_{i=1}^k}
\newcommand{\vov}{\Vert\omega\Vert} \newcommand{\Umy}{U_{\mu_i,y^i}}
\newcommand{\lamns}{\lambda_n^{^{\scriptstyle\sigma}}}
\newcommand{\chiomn}{\chi_{_{\Omega_n}}}
\newcommand{\ullim}{\underline{\lim}} \newcommand{\bsy}{\boldsymbol}
\newcommand{\mvb}{\mathversion{bold}} \newcommand{\la}{\lambda}
\newcommand{\La}{\Lambda} \newcommand{\va}{\varepsilon}
\newcommand{\be}{\beta} \newcommand{\al}{\alpha}
\newcommand{\dis}{\displaystyle} \newcommand{\R}{{\mathbb R}}

\newcommand{\N}{{\mathbb N}} \newcommand{\cF}{{\mathcal F}}
\newcommand{\gB}{{\mathfrak B}} \newcommand{\eps}{\epsilon}
\renewcommand\refname{参考文献}\renewcommand\figurename{图}
\usepackage[]{caption2} \renewcommand{\captionlabeldelim}{}
\begin{document}
\title{\huge{\bf{Linear algebra,Exercise 7}}} \author{\small{Luqing Ye\footnote{叶卢庆(1992---),E-mail:yeluqingmathematics@gmail.com}}}
\maketitle
\begin{exercise}[\href{http://www.math.ucla.edu/~tao/resource/general/115a.3.02f/assign7.pdf}{Tao}]
 Let $A$ and $B$ be similar $n\times n$ matrices.Show that $A$ and $B$
 have the same set of eigenvalues.
\end{exercise}
\begin{proof}
  $A$ and $B$ are similar matrices,so there exists an invertible
  $n\times n$ matrix $Q$ such that 
$$
A=Q^{-1}BQ.
$$
Suppose that $v_1$ is an eigenvector of $A$ with corresponding
eigenvalue $\lambda_1$,which means that
$$
\det(A-\lambda_1I)=0,
$$
So
$$
\det (Q^{-1}BQ-\lambda_1Q^{-1}Q)=\det Q^{-1}\det(B-\lambda_1I)\det Q=0,
$$
So
$$
\det(B-\lambda_1I)=0,
$$
So $\lambda_1$ is an eigenvalue of $B$.And vise versa.
\end{proof}
\end{document}

Let 
$$
A=
\begin{pmatrix}
  a_{11}&\cdots&a_{1n}\\
\vdots&\cdots&\vdots\\
a_{n1}&\cdots&a_{nn}
\end{pmatrix},B=
\begin{pmatrix}
  b_{11}&\cdots&b_{1n}\\
\vdots&\cdots&\vdots\\
b_{n1}&\cdots&b_{nn}
\end{pmatrix},Q=
\begin{pmatrix}
  c_{11}&\cdots&c_{1n}\\
\vdots&\cdots&\vdots\\
c_{n1}&\cdots&c_{nn}
\end{pmatrix}.
$$
