\documentclass[a4paper]{article}
\usepackage{amsmath,amsfonts,amsthm,amssymb} \usepackage{bm}
\usepackage{draftwatermark}
\SetWatermarkText{http://blog.sciencenet.cn/u/Yaleking}%设置水印文字
\SetWatermarkLightness{0.8}%设置水印亮度
\SetWatermarkScale{0.35}%设置水印大小
\usepackage{hyperref} \usepackage{geometry} \usepackage{yhmath}
\usepackage{pstricks-add} \usepackage{framed,mdframed}
\usepackage{graphicx,color} \usepackage{mathrsfs,xcolor}
\usepackage[all]{xy} \usepackage{fancybox} \usepackage{xeCJK}
\newtheorem*{theo}{定理} 
\newtheorem*{exa}{习题}
\newtheorem*{rem}{评论}
\newenvironment{theorem}
{\bigskip\begin{mdframed}\begin{theo}}
    {\end{theo}\end{mdframed}\bigskip} 
\newenvironment{remark}
{\bigskip\begin{mdframed}\begin{rem}}
    {\end{rem}\end{mdframed}\bigskip} 
\newenvironment{example}
{\bigskip\begin{mdframed}\begin{exa}}
    {\end{exa}\end{mdframed}\bigskip}
\geometry{left=2.5cm,right=2.5cm,top=2.5cm,bottom=2.5cm}
\setCJKmainfont[BoldFont=SimHei]{SimSun}
\newcommand{\D}{\displaystyle}\newcommand{\ri}{\Rightarrow}
\newcommand{\ds}{\displaystyle} \renewcommand{\ni}{\noindent}
\newcommand{\pa}{\partial} \newcommand{\Om}{\Omega}
\newcommand{\om}{\omega} \newcommand{\sik}{\sum_{i=1}^k}
\newcommand{\vov}{\Vert\omega\Vert} \newcommand{\Umy}{U_{\mu_i,y^i}}
\newcommand{\lamns}{\lambda_n^{^{\scriptstyle\sigma}}}
\newcommand{\chiomn}{\chi_{_{\Omega_n}}}
\newcommand{\ullim}{\underline{\lim}} \newcommand{\bsy}{\boldsymbol}
\newcommand{\mvb}{\mathversion{bold}} \newcommand{\la}{\lambda}
\newcommand{\La}{\Lambda} \newcommand{\va}{varepsilon}
\newcommand{\be}{\beta} \newcommand{\al}{\alpha}
\newcommand{\dis}{\displaystyle} \newcommand{\R}{{\mathbb R}}
\renewcommand{\today}{\number\year 年 \number\month 月 \number\day 日}
\newcommand{\N}{{\mathbb N}} \newcommand{\cF}{{\mathcal F}}
\newcommand{\gB}{{\mathfrak B}} \newcommand{\eps}{\epsilon}\newcommand{\op}{\operatorname}
\renewcommand\refname{参考文献}\renewcommand\figurename{图}
\usepackage[]{caption2} \renewcommand{\captionlabeldelim}{}
\begin{document}
\title{\huge{\bf{欧式空间中的线性变换是正交的当且仅当它把一组正交基变
      成一组正交基}}} \author{\small{叶卢
    庆\footnote{叶卢庆(1992---),男,杭州师范大学理学院数学与应用数学专业
      本科在读,E-mail:yeluqingmathematics@gmail.com}}}
\maketitle\ni
下面的定理来自张禾瑞等人编的《高等代数》第5版,定理8.3.2.
\begin{theorem}
设$V$是一个$n$维欧式空间.$\sigma$是$V$的一个线性变换.如果$\sigma$把$V$
的一组标准正交基仍旧变成$V$的一组标准正交基,那么$\sigma$是$V$的一个正
交变换.  
\end{theorem}
\begin{proof}[\bf{证明}]
 设$\alpha=(v_1,v_2,\cdots,v_n)$是$V$的一组有序标准正交基.则$V$中的任
 意两个向量$w_1,w_2$在$\alpha$下可以分别用坐标
 $(k_1,k_2,\cdots,k_n),(l_1,l_2,\cdots,l_n)$表示.现在我们来看
\begin{align*}
\langle w_1,w_2\rangle&=\langle k_1v_1+k_2v_2+\cdots+k_nv_n,l_1v_1+l_2v_2+\cdots+l_nv_n\rangle\\&=k_1l_1+k_2l_2+\cdots+k_nl_n.
\end{align*}
设$\sigma$把标准正交基$\alpha$变成另一组标准正交基
$\beta=(e_1,e_2,\cdots,e_n)$,则向量$w_1$在$\sigma$的作用下变成
$$
w_1'=k_1e_1+k_2e_2+\cdots+k_ne_n,
$$
向量$w_2$在$\sigma$的作用下变成
$$
w_2'=l_1e_1+l_2e_2+\cdots+l_ne_n.
$$
显然$\langle w_1',w_2'\rangle$仍然等于$k_1l_1+\cdots+\cdots+k_nl_n$.可
见变换$\sigma$保内积,于是$\sigma$是正交变换.
\end{proof}
\end{document}





























