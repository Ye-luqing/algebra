\documentclass[a4paper]{article}
\usepackage{amsmath,amsfonts,amsthm,amssymb}
\usepackage{bm}
\usepackage{draftwatermark}
\SetWatermarkText{http://blog.sciencenet.cn/u/Yaleking}%设置水印文字
\SetWatermarkLightness{0.8}%设置水印亮度
\SetWatermarkScale{0.35}%设置水印大小
\usepackage{hyperref}
\usepackage{geometry}
\usepackage{yhmath}
\usepackage{pstricks-add}
\usepackage{framed,mdframed}
\usepackage{graphicx,color} 
\usepackage{mathrsfs,xcolor} 
\usepackage[all]{xy}
\usepackage{fancybox} 
\usepackage{xeCJK}
\newtheorem{theo}{定理}
\newtheorem*{exe}{题目}
\newtheorem*{rem}{评论}
\newmdtheoremenv{lemma}{引理}
\newmdtheoremenv{corollary}{推论}
\newmdtheoremenv{example}{例}
\newenvironment{theorem}
{\bigskip\begin{mdframed}\begin{theo}}
    {\end{theo}\end{mdframed}\bigskip}
\newenvironment{exercise}
{\bigskip\begin{mdframed}\begin{exe}}
    {\end{exe}\end{mdframed}\bigskip}
\geometry{left=2.5cm,right=2.5cm,top=2.5cm,bottom=2.5cm}
\setCJKmainfont[BoldFont=SimHei]{SimSun}
\renewcommand{\today}{\number\year 年 \number\month 月 \number\day 日}
\newcommand{\D}{\displaystyle}\newcommand{\ri}{\Rightarrow}
\newcommand{\ds}{\displaystyle} \renewcommand{\ni}{\noindent}
\newcommand{\pa}{\partial} \newcommand{\Om}{\Omega}
\newcommand{\om}{\omega} \newcommand{\sik}{\sum_{i=1}^k}
\newcommand{\vov}{\Vert\omega\Vert} \newcommand{\Umy}{U_{\mu_i,y^i}}
\newcommand{\lamns}{\lambda_n^{^{\scriptstyle\sigma}}}
\newcommand{\chiomn}{\chi_{_{\Omega_n}}}
\newcommand{\ullim}{\underline{\lim}} \newcommand{\bsy}{\boldsymbol}
\newcommand{\mvb}{\mathversion{bold}} \newcommand{\la}{\lambda}
\newcommand{\La}{\Lambda} \newcommand{\va}{\varepsilon}
\newcommand{\be}{\beta} \newcommand{\al}{\alpha}
\newcommand{\dis}{\displaystyle} \newcommand{\R}{{\mathbb R}}
\newcommand{\N}{{\mathbb N}} \newcommand{\cF}{{\mathcal F}}
\newcommand{\gB}{{\mathfrak B}} \newcommand{\eps}{\epsilon}
\renewcommand\refname{参考文献}\renewcommand\figurename{图}
\usepackage[]{caption2} 
\renewcommand{\captionlabeldelim}{}
\begin{document}
\title{\huge{\bf{线性代数,习题16}}} \author{\small{叶卢
    庆\footnote{叶卢庆(1992---),男,杭州师范大学理学院数学与应用数学专业
      本科在读,E-mail:yeluqingmathematics@gmail.com}}}
\maketitle
\begin{exercise}
  设 $A$ 是矩阵
$$
A:=
\begin{pmatrix}
  0&1&0\\
-1&0&0\\
0&0&-1
\end{pmatrix}
$$
\begin{itemize}
\item 求一个复可逆矩阵 $Q$ 和复对角矩阵 $D$,使得 $A=QDQ^{-1}$.
\item 将 $A$ 分解成三个初等矩阵的积.
\end{itemize}
\end{exercise}
\begin{proof}[\textbf{证明}]
\begin{itemize}
\item 我们先求出矩阵 $A$ 的特征向量.我们来看 $A$ 的特征多项式
$$
\begin{vmatrix}
  -\lambda&1&0\\
-1&-\lambda&0\\
0&0&-1-\lambda
\end{vmatrix}=-(1+\lambda)(\lambda+i)(\lambda-i).
$$
可见,$A$ 有三个不同的复特征值,分别为 $-1,i,-i$.对应于这三个特征值分别
有三个线性无关的向量 $(0,0,1),(1,i,0),(1,-i,0)$.可见,$A$ 确实能对角化.设
$\alpha=((1,0,0),(0,1,0),(0,0,1))$ 是 $\mathbf{C}^3$ 的一组有序
基,$\beta=((0,0,1),(1,i,0),(1,-i,0))$ 是 $\mathbf{C}^3$ 的另一组基.
$$
A=[T]_{\alpha}^{\alpha}=[I]_{\beta}^{\alpha}[T]_{\beta}^{\beta}[I]_{\alpha}^{\beta}.
$$
而
$$
[I]_{\beta}^{\alpha}=
\begin{pmatrix}
  0&1&1\\
0&i&-i\\
1&0&0
\end{pmatrix} ,[I]_{\alpha}^{\beta}=
\begin{pmatrix}
  0&0&1\\
\frac{1}{2}&\frac{-i}{2}&0\\
\frac{1}{2}&\frac{i}{2}&0
\end{pmatrix}.
$$
可见,
$$
A=\begin{pmatrix}
  0&1&1\\
0&i&-i\\
1&0&0
\end{pmatrix}
\begin{pmatrix} 
  -1&0&0\\
0&i&0\\
0&0&-i
\end{pmatrix}\begin{pmatrix}
  0&0&1\\
\frac{1}{2}&\frac{-i}{2}&0\\
\frac{1}{2}&\frac{i}{2}&0
\end{pmatrix}.
$$
\item
$$
A=
\begin{pmatrix}
  0&1&0\\
  1&0&0\\
  0&0&1
\end{pmatrix}
\begin{pmatrix}
  -1&0&0\\
0&1&0\\
0&0&1
\end{pmatrix}
\begin{pmatrix}
  1&0&0\\
0&1&0\\
0&0&-1
\end{pmatrix}
$$
\end{itemize}
\end{proof}
\end{document}
