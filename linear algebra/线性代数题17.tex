\documentclass[a4paper]{article}
\usepackage{amsmath,amsfonts,amsthm,amssymb}
\usepackage{bm}
\usepackage{draftwatermark}
\SetWatermarkText{http://blog.sciencenet.cn/u/Yaleking}%设置水印文字
\SetWatermarkLightness{0.8}%设置水印亮度
\SetWatermarkScale{0.35}%设置水印大小
\usepackage{hyperref}
\usepackage{geometry}
\usepackage{yhmath}
\usepackage{pstricks-add}
\usepackage{framed,mdframed}
\usepackage{graphicx,color} 
\usepackage{mathrsfs,xcolor} 
\usepackage[all]{xy}
\usepackage{fancybox} 
\usepackage{xeCJK}
\newtheorem{theo}{定理}
\newtheorem*{exe}{题目}
\newtheorem*{rem}{评论}
\newmdtheoremenv{lemma}{引理}
\newmdtheoremenv{corollary}{推论}
\newmdtheoremenv{example}{例}
\newenvironment{theorem}
{\bigskip\begin{mdframed}\begin{theo}}
    {\end{theo}\end{mdframed}\bigskip}
\newenvironment{exercise}
{\bigskip\begin{mdframed}\begin{exe}}
    {\end{exe}\end{mdframed}\bigskip}
\geometry{left=2.5cm,right=2.5cm,top=2.5cm,bottom=2.5cm}
\setCJKmainfont[BoldFont=SimHei]{SimSun}
\renewcommand{\today}{\number\year 年 \number\month 月 \number\day 日}
\newcommand{\D}{\displaystyle}\newcommand{\ri}{\Rightarrow}
\newcommand{\ds}{\displaystyle} \renewcommand{\ni}{\noindent}
\newcommand{\pa}{\partial} \newcommand{\Om}{\Omega}
\newcommand{\om}{\omega} \newcommand{\sik}{\sum_{i=1}^k}
\newcommand{\vov}{\Vert\omega\Vert} \newcommand{\Umy}{U_{\mu_i,y^i}}
\newcommand{\lamns}{\lambda_n^{^{\scriptstyle\sigma}}}
\newcommand{\chiomn}{\chi_{_{\Omega_n}}}
\newcommand{\ullim}{\underline{\lim}} \newcommand{\bsy}{\boldsymbol}
\newcommand{\mvb}{\mathversion{bold}} \newcommand{\la}{\lambda}
\newcommand{\La}{\Lambda} \newcommand{\va}{\varepsilon}
\newcommand{\be}{\beta} \newcommand{\al}{\alpha}
\newcommand{\dis}{\displaystyle} \newcommand{\R}{{\mathbb R}}
\newcommand{\N}{{\mathbb N}} \newcommand{\cF}{{\mathcal F}}
\newcommand{\gB}{{\mathfrak B}} \newcommand{\eps}{\epsilon}
\renewcommand\refname{参考文献}\renewcommand\figurename{图}
\usepackage[]{caption2} 
\renewcommand{\captionlabeldelim}{}
\begin{document}
\title{\huge{\bf{线性代数,习题17}}} \author{\small{叶卢
    庆\footnote{叶卢庆(1992---),男,杭州师范大学理学院数学与应用数学专业
      本科在读,E-mail:yeluqingmathematics@gmail.com}}}
\maketitle
\begin{exercise}
  设 $f,g$ 为$[-1,1]$ 上的连续复值函数,满足 $\int_{-1}^1|f(x)|^2dx=9,\int_{-1}^1|g(x)|^2dx=16$.
  \begin{itemize}
  \item $\int_{-1}^1f(x)\overline{g(x)}dx$ 可能取什么值?
\item $\int_{-1}^1|f(x)+g(x)|^2dx$ 可能取到什么值?
  \end{itemize}
\end{exercise}
\begin{proof}[\textbf{证明}]
  \begin{itemize}
  \item 根据Cauchy-Schwartz 定理,
$$-12\leq|\int_{-1}^1f(x)\overline{g(x)}dx|\leq 12.$$
因此,$\int_{-1}^1f(x)\overline{g(x)}dx$ 可以取到 $\{a+bi|a,b\in
\mathbf{R},0\leq a^2+b^2\leq 144\}$ 中的值.
\item 
  \begin{align*}
    \int_{-1}^1|f(x)+g(x)|^2dx&=\int_{-1}^1(f(x)+g(x))(\overline{f(x)}+\overline{g(x)})dx\\&=\int_{-1}^1(|f(x)|^2+f(x)\overline{g(x)}+g(x)\overline{f(x)}+|g(x)|^2)dx\\&=\int_{-1}^1(f(x)\overline{g(x)}+g(x)\overline{f(x)})dx+25,
  \end{align*}
而 $1\leq\int_{-1}^1(f(x)\overline{g(x)}+g(x)\overline{f(x)})dx+25\leq 49$.
  \end{itemize}
\end{proof}
\end{document}
