\documentclass[a4paper]{article}
\usepackage{amsmath,amsfonts,amsthm,amssymb}
\usepackage{bm}
\usepackage{draftwatermark}
\SetWatermarkText{http://blog.sciencenet.cn/u/Yaleking}%设置水印文字
\SetWatermarkLightness{0.8}%设置水印亮度
\SetWatermarkScale{0.35}%设置水印大小
\usepackage{hyperref}
\usepackage{geometry}
\usepackage{yhmath}
\usepackage{pstricks-add}
\usepackage{framed,mdframed}
\usepackage{graphicx,color} 
\usepackage{mathrsfs,xcolor} 
\usepackage[all]{xy}
\usepackage{fancybox} 
\usepackage{xeCJK}
\newtheorem*{theo}{Theorem}
\newtheorem*{exe}{Exercise}
\newenvironment{theorem}
{\bigskip\begin{mdframed}\begin{theo}}
    {\end{theo}\end{mdframed}\bigskip}
\newenvironment{exercise}
{\bigskip\begin{mdframed}\begin{exe}}
    {\end{exe}\end{mdframed}\bigskip}
\geometry{left=2.5cm,right=2.5cm,top=2.5cm,bottom=2.5cm}
\setCJKmainfont[BoldFont=SimHei]{SimSun}
\newcommand{\D}{\displaystyle}\newcommand{\ri}{\Rightarrow}
\newcommand{\ds}{\displaystyle} \renewcommand{\ni}{\noindent}
\newcommand{\pa}{\partial} \newcommand{\Om}{\Omega}
\newcommand{\om}{\omega} \newcommand{\sik}{\sum_{i=1}^k}
\newcommand{\vov}{\Vert\omega\Vert} \newcommand{\Umy}{U_{\mu_i,y^i}}
\newcommand{\lamns}{\lambda_n^{^{\scriptstyle\sigma}}}
\newcommand{\chiomn}{\chi_{_{\Omega_n}}}
\newcommand{\ullim}{\underline{\lim}} \newcommand{\bsy}{\boldsymbol}
\newcommand{\mvb}{\mathversion{bold}} \newcommand{\la}{\lambda}
\newcommand{\La}{\Lambda} \newcommand{\va}{\varepsilon}
\newcommand{\be}{\beta} \newcommand{\al}{\alpha}
\newcommand{\dis}{\displaystyle} \newcommand{\R}{{\mathbb R}}
\newcommand{\N}{{\mathbb N}} \newcommand{\cF}{{\mathcal F}}
\newcommand{\gB}{{\mathfrak B}} \newcommand{\eps}{\epsilon}
\renewcommand\refname{参考文献}\renewcommand\figurename{图}
\usepackage[]{caption2} 
\renewcommand{\captionlabeldelim}{}
\begin{document}
\title{\huge{\bf{Linear algebra,Exercise 4}}} \author{\small{叶卢
    庆\footnote{Luqing
      Ye(1992---),E-mail:yeluqingmathematics@gmail.com}}}
\maketitle
\begin{exercise}\footnote{From
    http://www.math.ucla.edu/~tao/resource/general/115a.3.02f/practice.pdf}
  Consider the linear transformation $T:\mathbf{R}^3\to \mathbf{R}^3$
  defined by
$$
T \begin{pmatrix}
  x_1\\
  x_2\\
  x_3
\end{pmatrix}:=\begin{pmatrix}
  1&1&2\\
  2&1&3\\
  3&1&4
\end{pmatrix}\begin{pmatrix}
  x_1\\
  x_2\\
  x_3
\end{pmatrix}.
$$
What is the rank and nullity of $T$?
\end{exercise}
\begin{proof}[\textbf{Solve}]
  Let
  $\mathbf{v}_1=(1,0,0),\mathbf{v}_2=(0,1,0),\mathbf{v}_3=(0,0,1)$.Then
  $\{\mathbf{v}_1,\mathbf{v}_2,\mathbf{v}_3\}$ is a basis of
  $\mathbf{R}^3$.
$$
\begin{cases}
  T(\mathbf{v}_1)=(1,2,3)=\mathbf{v}_1+2\mathbf{v}_2+3\mathbf{v}_3,\\
  T(\mathbf{v}_2)=(1,1,1)=\mathbf{v}_1+\mathbf{v}_2+\mathbf{v}_3,\\
  T(\mathbf{v}_3)=(2,3,4)=2\mathbf{v}_1+3\mathbf{v}_2+4\mathbf{v}_3.
\end{cases}
$$
So
$$
\begin{cases}
  T(\mathbf{v}_1)=\mathbf{v}_1+2\mathbf{v}_2+3\mathbf{v}_3,\\
T(\mathbf{v}_1-\mathbf{v}_2)=\mathbf{v}_2+2\mathbf{v}_3,\\
T(2\mathbf{v}_1-\mathbf{v}_3)=\mathbf{v}_2+2\mathbf{v}_3.
\end{cases}
$$
So
$$
\begin{cases}
  T(\mathbf{v}_1)=\mathbf{v}_1+2\mathbf{v}_2+3\mathbf{v}_3,\\
T(\mathbf{v}_1-\mathbf{v}_2)=\mathbf{v}_2+2\mathbf{v}_3,\\
T(-\mathbf{v}_1-\mathbf{v}_2+\mathbf{v}_3)=\mathbf{0}.
\end{cases}
$$
So the rank of $T$ is $2$,the
nullity of $T$ is $1$.
\end{proof}
\end{document}
