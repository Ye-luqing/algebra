\documentclass[a4paper]{article}
\usepackage{amsmath,amsfonts,amsthm,amssymb} \usepackage{bm}
\usepackage{draftwatermark,euler}
\SetWatermarkText{http://blog.sciencenet.cn/u/Yaleking}%设置水印文字
\SetWatermarkLightness{0.8}%设置水印亮度
\SetWatermarkScale{0.35}%设置水印大小
\usepackage{hyperref} \usepackage{geometry} \usepackage{yhmath}
\usepackage{pstricks-add} \usepackage{framed,mdframed}
\usepackage{graphicx,color} \usepackage{mathrsfs,xcolor}
\usepackage[all]{xy} \usepackage{fancybox} \usepackage{xeCJK}
\newtheorem*{theo}{定理} 
\newtheorem*{exe}{Exercise}
\newenvironment{theorem}
{\bigskip\begin{mdframed}\begin{theo}}
    {\end{theo}\end{mdframed}\bigskip} 
\newenvironment{exercise}
{\bigskip\begin{mdframed}\begin{exe}}
    {\end{exe}\end{mdframed}\bigskip}
\geometry{left=2.5cm,right=2.5cm,top=2.5cm,bottom=2.5cm}
\setCJKmainfont[BoldFont=SimHei]{SimSun}
\newcommand{\D}{\displaystyle}\newcommand{\ri}{\Rightarrow}
\newcommand{\ds}{\displaystyle} \renewcommand{\ni}{\noindent}
\newcommand{\pa}{\partial} \newcommand{\Om}{\Omega}
\newcommand{\om}{\omega} \newcommand{\sik}{\sum_{i=1}^k}
\newcommand{\vov}{\Vert\omega\Vert} \newcommand{\Umy}{U_{\mu_i,y^i}}
\newcommand{\lamns}{\lambda_n^{^{\scriptstyle\sigma}}}
\newcommand{\chiomn}{\chi_{_{\Omega_n}}}
\newcommand{\ullim}{\underline{\lim}} \newcommand{\bsy}{\boldsymbol}
\newcommand{\mvb}{\mathversion{bold}} \newcommand{\la}{\lambda}
\newcommand{\La}{\Lambda} \newcommand{\va}{\varepsilon}
\newcommand{\be}{\beta} \newcommand{\al}{\alpha}
\newcommand{\dis}{\displaystyle} \newcommand{\R}{{\mathbb R}}

\newcommand{\N}{{\mathbb N}} \newcommand{\cF}{{\mathcal F}}
\newcommand{\gB}{{\mathfrak B}} \newcommand{\eps}{\epsilon}
\renewcommand\refname{参考文献}\renewcommand\figurename{图}
\usepackage[]{caption2} \renewcommand{\captionlabeldelim}{}
\begin{document}


\title{\huge{\bf{Linear algebra,Exercise 8}}} \author{\small{Luqing Ye\footnote{叶卢庆(1992---),E-mail:yeluqingmathematics@gmail.com}}}
\maketitle
\begin{exercise}[\href{http://www.math.ucla.edu/~tao/resource/general/115a.3.02f/assign7.pdf}{Tao}]
For this question,the field of scalars will be the complex numbers
instead of the reals(i.e,all matrices,etc.are allowed to have complex
entries).Let $\theta$ be a real number,and let $A$ be the $2\times 2$
rotational matrix
$$
A=\begin{pmatrix}
  \cos\theta&-\sin\theta\\
\sin\theta&\cos\theta
\end{pmatrix}.
$$
\begin{itemize}
\item Show that $A$ has eigenvalues $e^{i\theta},e^{-i\theta}$.What are the
eigenvectors corresponding to $e^{i\theta}$ and $e^{-i\theta}$?
\item Write $A=QDQ^{-1}$ for some invertible matrix $Q$ and diagonal
  matrix $D$. 
\end{itemize}
\end{exercise}
\begin{proof}
  \begin{itemize}
  \item 
$$
\begin{pmatrix}
  \cos\theta&-\sin\theta\\
\sin\theta&\cos\theta  
\end{pmatrix}
\begin{pmatrix}
  x\\
y
\end{pmatrix}=\lambda
\begin{pmatrix}
  x\\
y\\
\end{pmatrix},
$$
so
$$
\begin{pmatrix}
  \cos\theta-\lambda&-\sin\theta\\
\sin\theta&\cos\theta-\lambda
\end{pmatrix}
\begin{pmatrix}
  x\\
y\\
\end{pmatrix}=
\begin{pmatrix}
  0\\
0
\end{pmatrix}
,
$$
where $(x,y)\neq (0,0)$.So
$$
(\cos\theta-\lambda)^2+\sin^2\theta=0\iff
\lambda^{2}-2\lambda\cos\theta+1=0\iff \lambda=\cos\theta\pm i\sin\theta.
$$
When $\lambda=\cos\theta+i\sin\theta$,then
$$
\begin{pmatrix}
-  i\sin\theta&-\sin\theta\\
\sin\theta&-i\sin\theta
\end{pmatrix}
\begin{pmatrix}
  x\\
y
\end{pmatrix}=
\begin{pmatrix}
  0\\
0
\end{pmatrix}.
$$
So when $\theta=\pi k,k\in \mathbf{Z}$,the eigenvector
corresponding to
$e^{i\theta}$ is  arbitary,otherwise,the eigenvector is in the form
of $(iy,y),y\in \mathbf{C}\backslash\{0\}$.\\

When $\lambda=\cos\theta-i\sin\theta$,then $ix\sin\theta -y\sin\theta
=0$.When $\theta=\pi k,k\in \mathbf{Z}$,then the eigenvector
corresponding to  $e^{i\theta}$ is arbitrary.Otherwise,the eigenvector
is in the form of $(x,ix),x\in \mathbf{C}\backslash\{0\}$.
\item 
Denote the linear transformation corresponding to the matrix
$A$ by $L_A$. The eigenvector of $A$,$(i,1)$ and $(1,i)$,are linearly independent in
$\mathbf{C}$.Let $\alpha=((i,1),(1,i))=(w_1,w_2)$ be an ordered basis,let
$\beta=((1,0),(0,1))=(v_1,v_2)$ be another ordered basis.Then 
$$ 
[L_{A}]_{\alpha}^{\alpha}=
\begin{pmatrix}
  e^{i\theta}&0\\
0&e^{-i\theta}
\end{pmatrix}.
$$
And
$$
\begin{cases}
  w_1=iv_1+v_2,\\
w_2=v_1+iv_2.
\end{cases}
$$
So
$$
[I]_{\alpha}^{\beta}=
\begin{pmatrix}
  i&1\\
1&i
\end{pmatrix},
[I]_{\beta}^{\alpha}=
\begin{pmatrix}
  \frac{-i}{2}&\frac{1}{2}\\
\frac{1}{2}&\frac{-i}{2}
\end{pmatrix}.
$$
So
$$
A=[L_A]_{\beta}^{\beta}=[I]_{\alpha}^{\beta}[L_A]_{\alpha}^{\alpha}[I]_{\beta}^{\alpha}.
$$
Done.
  \end{itemize}
\end{proof}
\end{document}
