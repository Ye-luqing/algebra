\documentclass[a4paper]{article}
\usepackage{amsmath,amsfonts,amsthm,amssymb}
\usepackage{bm}
\usepackage{draftwatermark}
\SetWatermarkText{http://blog.sciencenet.cn/u/Yaleking}%设置水印文字
\SetWatermarkLightness{0.8}%设置水印亮度
\SetWatermarkScale{0.35}%设置水印大小
\usepackage{hyperref}
\usepackage{geometry}
\usepackage{yhmath}
\usepackage{pstricks-add}
\usepackage{framed,mdframed}
\usepackage{graphicx,color} 
\usepackage{mathrsfs,xcolor} 
\usepackage[all]{xy}
\usepackage{fancybox} 
\usepackage{xeCJK}
\newtheorem*{theo}{Theorem}
\newtheorem*{exe}{Exercise}
\newenvironment{theorem}
{\bigskip\begin{mdframed}\begin{theo}}
    {\end{theo}\end{mdframed}\bigskip}
\newenvironment{exercise}
{\bigskip\begin{mdframed}\begin{exe}}
    {\end{exe}\end{mdframed}\bigskip}
\geometry{left=2.5cm,right=2.5cm,top=2.5cm,bottom=2.5cm}
\setCJKmainfont[BoldFont=SimHei]{SimSun}
\newcommand{\D}{\displaystyle}\newcommand{\ri}{\Rightarrow}
\newcommand{\ds}{\displaystyle} \renewcommand{\ni}{\noindent}
\newcommand{\pa}{\partial} \newcommand{\Om}{\Omega}
\newcommand{\om}{\omega} \newcommand{\sik}{\sum_{i=1}^k}
\newcommand{\vov}{\Vert\omega\Vert} \newcommand{\Umy}{U_{\mu_i,y^i}}
\newcommand{\lamns}{\lambda_n^{^{\scriptstyle\sigma}}}
\newcommand{\chiomn}{\chi_{_{\Omega_n}}}
\newcommand{\ullim}{\underline{\lim}} \newcommand{\bsy}{\boldsymbol}
\newcommand{\mvb}{\mathversion{bold}} \newcommand{\la}{\lambda}
\newcommand{\La}{\Lambda} \newcommand{\va}{\varepsilon}
\newcommand{\be}{\beta} \newcommand{\al}{\alpha}
\newcommand{\dis}{\displaystyle} \newcommand{\R}{{\mathbb R}}
\newcommand{\N}{{\mathbb N}} \newcommand{\cF}{{\mathcal F}}
\newcommand{\gB}{{\mathfrak B}} \newcommand{\eps}{\epsilon}
\renewcommand\refname{参考文献}\renewcommand\figurename{图}
\usepackage[]{caption2} 
\renewcommand{\captionlabeldelim}{}
\begin{document}
\title{\huge{\bf{Linear algebra,Exercise 3}}} \author{\small{叶卢
    庆\footnote{Luqing
      Ye(1992---),E-mail:yeluqingmathematics@gmail.com}}}
\maketitle
\begin{exercise}\footnote{This exercise is from
    http://www.math.ucla.edu/~tao/resource/general/115a.3.02f/practice.pdf}
  Let $V_1,V_2,V_3,V_4$ be vector spaces such that
$$
\dim (V_1)=8,\dim (V_2)=5,\dim (V_3)=7,\dim (V_4)=6.
$$
Let $T_1:V_1\to V_2$,$T_2:V_2\to V_3$,and $T_3:V_3\to V_4$ be linear
transformations.Let $T=T_3T_2T_1$ be their composition.Prove that $T$
is not surjective.
\end{exercise}
\begin{proof}
  Prove by contradiction.If $T$ is surjective,then $T_3$ must be
  surjective.Then $N(T_3)=7-6=1$,which means that there exists a basis
  $\{v_1,\cdots,v_6\}$ of $V_3$ such that there is a bijection $T_3'$
  from $Span (\{v_1,\cdots,v_6\})$ to $V_4$,where for all $x\in
  Span(\{v_1,\cdots,v_6\})$,$T_3'(x)=T_3(x)$.\\

So there is a surjection from $V_2$ to $V_3$.But $\dim (V_2)<\dim
(V_3)$,which leads to contradiction.
\end{proof}
\end{document}
