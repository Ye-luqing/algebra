\documentclass[twoside,11pt]{article} 
\usepackage{amsmath,amsfonts,bm}
\usepackage{hyperref}
\usepackage{amsthm} 
\usepackage{amssymb}
\usepackage{framed,mdframed}
\usepackage{graphicx,color} 
\usepackage{mathrsfs,xcolor} 
\usepackage[all]{xy}
\usepackage{fancybox} 
% \usepackage{CJKutf8}
\usepackage{xeCJK}
\newtheorem{theorem}{定理}
\newtheorem*{timu}{题目}
\newtheorem{lemma}{引理}
\newtheorem{corollary}{推论}
\newtheorem*{exercise}{习题}
\newtheorem*{example}{例}
\renewcommand{\today}{\number\year 年 \number\month 月 \number\day 日}
\setCJKmainfont[BoldFont=Adobe Heiti Std R]{Adobe Song Std L}
% \usepackage{latexdef}
\def\ZZ{\mathbb{Z}} \topmargin -0.40in \oddsidemargin 0.08in
\evensidemargin 0.08in \marginparwidth 0.00in \marginparsep 0.00in
\textwidth 16cm \textheight 24cm \newcommand{\D}{\displaystyle}
\newcommand{\ds}{\displaystyle} \renewcommand{\ni}{\noindent}
\newcommand{\pa}{\partial} \newcommand{\Om}{\Omega}
\newcommand{\om}{\omega} \newcommand{\sik}{\sum_{i=1}^k}
\newcommand{\vov}{\Vert\omega\Vert} \newcommand{\Umy}{U_{\mu_i,y^i}}
\newcommand{\lamns}{\lambda_n^{^{\scriptstyle\sigma}}}
\newcommand{\chiomn}{\chi_{_{\Omega_n}}}
\newcommand{\ullim}{\underline{\lim}} \newcommand{\bsy}{\boldsymbol}
\newcommand{\mvb}{\mathversion{bold}} \newcommand{\la}{\lambda}
\newcommand{\La}{\Lambda} \newcommand{\va}{\varepsilon}
\newcommand{\be}{\beta} \newcommand{\al}{\alpha}
\newcommand{\dis}{\displaystyle} \newcommand{\R}{{\mathbb R}}
\newcommand{\N}{{\mathbb N}} \newcommand{\cF}{{\mathcal F}}
\newcommand{\gB}{{\mathfrak B}} \newcommand{\eps}{\epsilon}
\renewcommand\refname{参考文献} \def \qed {\hfill \vrule height6pt
  width 6pt depth 0pt} \topmargin -0.40in \oddsidemargin 0.08in
\evensidemargin 0.08in \marginparwidth0.00in \marginparsep 0.00in
\textwidth 15.5cm \textheight 24cm \pagestyle{myheadings}
\markboth{\rm \centerline{}} {\rm \centerline{}}
\begin{document}
\title{\huge{\bf{利用对称多项式基本定理来证明一个不等式}}} \author{\small{叶卢
    庆\footnote{叶卢庆(1992---),男,杭州师范大学理学院数学与应用数学专业
      本科在读,E-mail:h5411167@gmail.com}}\\{\small{杭州师范大学理学院,浙
      江~杭州~310036}}} 
\maketitle
\vspace{30pt}
\begin{timu}
若 $x>0,y>0,z>0$,求证
$$
\frac{x^3}{x+y}+\frac{y^3}{y+z}+\frac{z^3}{z+x}\geq \frac{xy+yz+zx}{2}
$$
\end{timu}
\begin{proof}[\textbf{证明}]
我们只用证明
  \begin{align*}
    \frac{x^3(y+z)(z+x)+y^3(x+y)(z+x)+z^3(x+y)(y+z)}{(x+y)(y+z)(z+x)}\geq \frac{xy+yz+zx}{2}.
  \end{align*}
令
$$
\begin{cases}
  \sigma_1=x+y+z,\\
\sigma_2=xy+yz+zx,\\
\sigma_3=xyz.
\end{cases},
$$
易得
$$
(x+y)(y+z)(z+x)=\sigma_1\sigma_2-\sigma_3,
$$
且
$$
x^3(y+z)(z+x)+y^3(x+y)(z+x)+z^3(x+y)(y+z)=\sigma_1^3-3\sigma_1\sigma_2+3\sigma_3.
$$
则我们只用证明
$$
2(\sigma_1^3-3\sigma_1\sigma_2+3\sigma_3)\geq \sigma_1\sigma_2^2-\sigma_2\sigma_3.
$$
我做不了了,还是先放着吧.
\end{proof}
\end{document}








