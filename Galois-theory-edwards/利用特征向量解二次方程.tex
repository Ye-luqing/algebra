\documentclass[a4paper]{article} 
\usepackage{amsmath,amsfonts,bm}
\usepackage{hyperref}
\usepackage{amsthm} 
\usepackage{geometry}
\usepackage{amssymb}
\usepackage{pstricks-add}
\usepackage{framed,mdframed}
\usepackage{graphicx,color} 
\usepackage{mathrsfs,xcolor} 
\usepackage[all]{xy}
\usepackage{fancybox} 
\usepackage{xeCJK}
\newtheorem*{theorem}{定理}
\newtheorem*{lemma}{引理}
\newtheorem*{corollary}{推论}
\newtheorem*{exercise}{习题}
\newtheorem*{example}{例}
\geometry{left=2.5cm,right=2.5cm,top=2.5cm,bottom=2.5cm}
\setCJKmainfont[BoldFont=Adobe Heiti Std R]{Adobe Song Std L}
\renewcommand{\today}{\number\year 年 \number\month 月 \number\day 日}
\newcommand{\D}{\displaystyle}\newcommand{\ri}{\Rightarrow}
\newcommand{\ds}{\displaystyle} \renewcommand{\ni}{\noindent}
\newcommand{\pa}{\partial} \newcommand{\Om}{\Omega}
\newcommand{\om}{\omega} \newcommand{\sik}{\sum_{i=1}^k}
\newcommand{\vov}{\Vert\omega\Vert} \newcommand{\Umy}{U_{\mu_i,y^i}}
\newcommand{\lamns}{\lambda_n^{^{\scriptstyle\sigma}}}
\newcommand{\chiomn}{\chi_{_{\Omega_n}}}
\newcommand{\ullim}{\underline{\lim}} \newcommand{\bsy}{\boldsymbol}
\newcommand{\mvb}{\mathversion{bold}} \newcommand{\la}{\lambda}
\newcommand{\La}{\Lambda} \newcommand{\va}{\varepsilon}
\newcommand{\be}{\beta} \newcommand{\al}{\alpha}
\newcommand{\dis}{\displaystyle} \newcommand{\R}{{\mathbb R}}
\newcommand{\N}{{\mathbb N}} \newcommand{\cF}{{\mathcal F}}
\newcommand{\gB}{{\mathfrak B}} \newcommand{\eps}{\epsilon}
\renewcommand\refname{参考文献}
\begin{document}
\title{\huge{\bf{利用特征向量求二次方程}}} \author{\small{叶卢
    庆}}
\maketitle
易得矩阵
$$
A=\begin{pmatrix}
  -p&-q\\
1&0
\end{pmatrix}
$$
的特征方程为
\begin{equation}\label{eq:0}
\lambda^2+p\lambda+q=0.
\end{equation}
其中 $p,q\in \mathbf{C}$.设该矩阵特征值分别为 $\lambda_1,\lambda_2$,且$\lambda_1,\lambda_2\in \mathbf{R}$,且
$$
A(\mathbf{v_1})=\lambda_1\mathbf{v_1},A(\mathbf{v_2})=\lambda_2\mathbf{v_2}.
$$
易得 $\mathbf{v_1}$ 和 $\mathbf{v_2}$ 是 $A$ 的特征向量.令
\begin{equation}\label{eq:1}
\mathbf{v_1}=(\frac{\lambda_1}{\sqrt{1+\lambda_1^2}},\frac{1}{\sqrt{1+\lambda_1^2}}),\mathbf{v_2}=(\frac{\lambda_2}{\sqrt{1+\lambda_2^2}},\frac{1}{\sqrt{1+\lambda_2^2}}),
\end{equation}
可得 $|\mathbf{v_1}|=|\mathbf{v_2}|=1$.由韦达定理易得
$$
\frac{\mathbf{v_1}+\mathbf{v_2}}{2}=\left(\frac{1}{2}\sqrt{\frac{p^2-2q+2q^2}{(q-1)^2+p^2}+\frac{2q}{\sqrt{(q-1)^2+p^2}}},\frac{1}{2}\sqrt{\frac{2}{\sqrt{(q-1)^2+p^2}}+\frac{2+p^2-2q}{(q-1)^2+p^2}}\right).
$$
由于 $\frac{\mathbf{v_1-v_2}}{2}$ 与 $\frac{\mathbf{v_1+v_2}}{2}$ 垂直,因此可得
$$
\frac{\mathbf{v_1}-\mathbf{v_2}}{2}=k\left(\frac{1}{2}\sqrt{\frac{2}{\sqrt{(q-1)^2+p^2}}+\frac{2+p^2-2q}{(q-1)^2+p^2}},-\frac{1}{2}\sqrt{\frac{p^2-2q+2q^2}{(q-1)^2+p^2}+\frac{2q}{\sqrt{(q-1)^2+p^2}}}\right),
$$
其中 $k\in \mathbf{R}$.由于
$$
\left|\frac{\mathbf{v_1}-\mathbf{v_2}}{2}\right|^2+\left|\frac{\mathbf{v_1}+\mathbf{v_2}}{2}\right|^2=|\mathbf{v_1}|^2=1,
$$
因此便可以解得 $k$,因此我们解得了$\frac{\mathbf{v_1}+\mathbf{v_2}}{2}$ 和$\frac{\mathbf{v_1}-\mathbf{v_2}}{2}$,因此可以解得$\mathbf{v_1},\mathbf{v_2}$.$\mathbf{v_1},\mathbf{v_2}$ 既解,那么由表达式 \eqref{eq:1},$\lambda_1,\lambda_2$ 也能解出.这样就得到了一元二次方程\eqref{eq:0} 的两个实根 $\lambda_1,\lambda_2$.
\end{document}








