\documentclass[a4paper]{article} 
\usepackage{amsmath,amsfonts,bm}
\usepackage{hyperref}
\usepackage{amsthm} 
\usepackage{geometry}
\usepackage{amssymb}
\usepackage{pstricks-add}
\usepackage{framed,mdframed}
\usepackage{graphicx,color} 
\usepackage{mathrsfs,xcolor} 
\usepackage[all]{xy}
\usepackage{fancybox} 
% \usepackage{CJKutf8}
\usepackage{xeCJK}
\newtheorem{theorem}{定理}
\newtheorem{lemma}{引理}
\newtheorem{corollary}{推论}
\newtheorem*{exercise}{习题}
\newtheorem{example}{例}
\geometry{left=2.5cm,right=2.5cm,top=2.5cm,bottom=2.5cm}
\setCJKmainfont[BoldFont=Adobe Heiti Std R]{Adobe Song Std L}
\renewcommand{\today}{\number\year 年 \number\month 月 \number\day 日}
\newcommand{\D}{\displaystyle}
\newcommand{\ds}{\displaystyle} \renewcommand{\ni}{\noindent}
\newcommand{\pa}{\partial} \newcommand{\Om}{\Omega}
\newcommand{\om}{\omega} \newcommand{\sik}{\sum_{i=1}^k}
\newcommand{\vov}{\Vert\omega\Vert} \newcommand{\Umy}{U_{\mu_i,y^i}}
\newcommand{\lamns}{\lambda_n^{^{\scriptstyle\sigma}}}
\newcommand{\chiomn}{\chi_{_{\Omega_n}}}
\newcommand{\ullim}{\underline{\lim}} \newcommand{\bsy}{\boldsymbol}
\newcommand{\mvb}{\mathversion{bold}} \newcommand{\la}{\lambda}
\newcommand{\La}{\Lambda} \newcommand{\va}{\varepsilon}
\newcommand{\be}{\beta} \newcommand{\al}{\alpha}
\newcommand{\dis}{\displaystyle} \newcommand{\R}{{\mathbb R}}
\newcommand{\N}{{\mathbb N}} \newcommand{\cF}{{\mathcal F}}
\newcommand{\gB}{{\mathfrak B}} \newcommand{\eps}{\epsilon}
\renewcommand\refname{参考文献}
\begin{document}
\title{\huge{\bf{如何解三次方程}}} \author{\small{叶卢庆\footnote{叶卢
      庆(1992---),男,杭州师范大学理学院数学与应用数学专业本科在
      读,E-mail:h5411167@gmail.com}}\\{\small{杭州师范大学理学
      院,浙江~杭州~310036}}}
\maketitle
在试图解三次方程之前,我们来看二次方程是如何解的.设二次方程
$$
x^2+bx+c=0
$$
的根为 $x_1,x_2$,根据韦达定理,我们有
$$
x_1+x_2=-b,x_1x_2=c.
$$
为了求解 $x_1,x_2$,我们只需要求解
出 $\alpha_1x_1+\alpha_2x_2$,其中$\alpha_1,\alpha_2$ 为两个非零常
数,且$\alpha_1:\alpha_2\neq 1:1$,然后联立方程组
$$
\begin{cases}
  x_1+x_2=-b,\\
  \alpha_1x_1+\alpha_2x_2=k,
\end{cases}
$$
便可解得 $x_1,x_2$.显然,$\alpha_1x_1+\alpha_2x_2$ 不是关于 $x_1,x_2$的
对称多项式,但是
$$
(\alpha_1x_1+\alpha_2x_2)(\alpha_2x_1+\alpha_1x_2)
$$
是关于 $x_1,x_2$ 的对称多项式,根据对称多项式基本定理,
$$
(\alpha_1x_1+\alpha_2x_2)(\alpha_2x_1+\alpha_1x_2)
$$
可以被 $\alpha_1,\alpha_2,b,c$ 所表示,具体的,
$$
(\alpha_1x_1+\alpha_2x_2)(\alpha_2x_1+\alpha_1x_2)=\alpha_1\alpha_2(b^2-2c)+(\alpha_1^2+\alpha_2^2)c.
$$
我们希望$\alpha_1x_1+\alpha_2x_2$ 和 $\alpha_2x_1+\alpha_1x_2$ 之间存在
如下关系:
$$
\alpha_1x_1+\alpha_2x_2=p(\alpha_2x_1+\alpha_1x_2),
$$
其中 $p$ 是常数,则 $p^2=1$,解得 $p=1$ 或者 $-1$.当
$p=1$时,$\alpha_1=\alpha_2$,与 $\alpha_1:\alpha_2=1:1$ 矛盾.当 $p=-1$
时,可得 $\alpha_1=-\alpha_2$,则我们得到
$$
-\alpha_1^2(x_1-x_2)^2=-\alpha_1^2b^2+4\alpha_1^2c.
$$
解得
$$
x_1-x_2=\pm\sqrt{b^2-4c}.
$$
然后联立方程组
$$
\begin{cases}
  x_1+x_2=-b,\\
  x_1-x_2=\pm \sqrt{b^2-4c}
\end{cases}
$$
可以解得 $x_1,x_2$.\\

下面我们仿照上面的方法,看怎么解三次方程
$$
x^3+bx^2+cx+d=0.
$$
设该三次方程的根为 $x_1,x_2,x_3$,根据韦达定理,
$$
\begin{cases}
  x_1+x_2+x_3=-b,\\
  x_1x_2+x_2x_3+x_3x_1=c\\
  x_1x_2x_3=-d.
\end{cases}
$$
我们希望存在
$$
\begin{cases}
  \alpha_1x_1+\alpha_2x_2+\alpha_3x_3=k_1,\\
  \beta_1x_1+\beta_2x_2+\beta_3x_3=k_2.\\
\end{cases}
$$
其中 $k_1,k_2,\alpha_1,\alpha_2,\alpha_3,\beta_1,\beta_2,\beta_3$ 都是
常数.我们希望这两条方程联立 $x_1+x_2+x_3=-b$ 形成的三元齐次线性方程组有
唯一解.显然,
\begin{equation}\label{eq:1}
  (\alpha_1x_1+\alpha_2x_2+\alpha_3x_3)(\alpha_1x_1+\alpha_3x_2+\alpha_2x_3)(\alpha_3x_1+\alpha_2x_2+\alpha_1x_3)(\alpha_2x_1+\alpha_1x_2+\alpha_3x_3)(\alpha_3x_1+\alpha_1x_2+\alpha_2x_3)(\alpha_2x_1+\alpha_3x_2+\alpha_1x_3)
\end{equation}
是关于 $x_1,x_2,x_3$ 的对称多项式,根据对称多项式基本定
理,式\eqref{eq:1} 可以表达为关于 $b,c,d$ 的
多项式,具体怎么表达这里就从略了,因为太麻烦,反正能表达出来就是了,我们只
把 \eqref{eq:1} 表达为 $f(b,c,d)$.然后,对
\eqref{eq:1} 式中的6个因式进行分组,分为两组,分别为
\begin{itemize}
\item $\alpha_1x_1+\alpha_2x_2+\alpha_3x_3$
\item $\alpha_3x_1+\alpha_1x_2+\alpha_2x_3$
\item $\alpha_2x_1+\alpha_3x_2+\alpha_1x_3$
\end{itemize}
以及
\begin{itemize}
\item $\alpha_3x_1+\alpha_2x_2+\alpha_1x_3$
\item $\alpha_1x_1+\alpha_3x_2+\alpha_2x_3$
\item $\alpha_2x_1+\alpha_1x_2+\alpha_3x_3$
\end{itemize}
分组依据是每一组的系数都进行轮换.然后令
$$
\alpha_1x_1+\alpha_2x_2+\alpha_3x_3=p(\alpha_3x_1+\alpha_1x_2+\alpha_2x_3)=p^2(\alpha_2x_1+\alpha_3x_2+\alpha_1x_3).
$$
以及
$$
\alpha_3x_1+\alpha_2x_2+\alpha_1x_3=q(\alpha_1x_1+\alpha_3x_2+\alpha_2x_3)=q^2(\alpha_2x_1+\alpha_1x_2+\alpha_3x_3).
$$
解得 $p^3=1$ 以及 $q^3=1$.当 $p,q=1$ 时,显然是没有什么意思的,因为此时
$\alpha_1=\alpha_2=\alpha_3$,这样就无法与 $x_1+x_2+x_3=-b$ 联立起来解
得方程组.当 $p=\omega_3$ ,其中 $\omega_3$ 是三次单位根 $e^{\frac{2\pi
    i}{3}}$ 时,得到
$$
\alpha_1=\omega_3\alpha_3=\omega_3^2\alpha_2,
$$
此时,$q=\omega_3^{-1}$.而当 $p=\omega_3^{-1}$ 时,解得
$$
\alpha_1=\omega_3^{-1}\alpha_3=\omega_3^{-2}\alpha_2,
$$
此时 $q=\omega_3$.因此由对称性,不妨令
$p=\omega_3$,$q=\omega_3^{-1}$.令
$\alpha_1x_1+\alpha_2x_2+\alpha_3x_3=U$,令
$\alpha_3x_1+\alpha_2x_2+\alpha_1x_3=V$,于是我们有
$$
U^3V^3=f(b,c,d).
$$
易得
\begin{align*}
&(\alpha_1x_1+\alpha_2x_2+\alpha_3x_3)(\alpha_3x_1+\alpha_1x_2+\alpha_2x_3)(\alpha_2x_1+\alpha_3x_2+\alpha_1x_3)\\&+(\alpha_3x_1+\alpha_2x_2+\alpha_1x_3)(\alpha_1x_1+\alpha_3x_2+\alpha_2x_3)(\alpha_2x_1+\alpha_1x_2+\alpha_3x_3)=U^3+V^3
\end{align*}
也是关于 $x_1,x_2,x_3$ 的对称多项式,因此根据对称多项式基本定理
$U^3+V^3$ 也能表达关于 $b,c,d$ 的多项式
$g(b,c,d)$.因此我们得到
$$
\begin{cases}
  U^3V^3=f(b,c,d),\\
U^3+V^3=g(b,c,d).
\end{cases}
$$
因此可以解得 $U^3,V^3$,进一步解得 $U,V$.然后联立方程组
$$
\begin{cases}
  x_1+x_2+x_3=-b,\\
\alpha_1x_1+\alpha_2x_2+\alpha_3x_3=U,\\
\alpha_3x_1+\alpha_2x_2+\alpha_3x_1=V.
\end{cases}
$$
解出 $x_1,x_2,x_3$ 即可(方程组中的 $\alpha_1,\alpha_2,\alpha_3$ 会与
$U,V$ 表达式中的 $\alpha_1,\alpha_2,\alpha_3$ 约去.).
\end{document}








