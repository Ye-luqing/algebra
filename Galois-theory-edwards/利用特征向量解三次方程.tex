\documentclass[a4paper]{article} 
\usepackage{amsmath,amsfonts,bm}
\usepackage{hyperref}
\usepackage{amsthm} 
\usepackage{geometry}
\usepackage{amssymb}
\usepackage{pstricks-add}
\usepackage{framed,mdframed}
\usepackage{graphicx,color} 
\usepackage{mathrsfs,xcolor} 
\usepackage[all]{xy}
\usepackage{fancybox} 
\usepackage{xeCJK}
\newtheorem*{theorem}{定理}
\newtheorem*{lemma}{引理}
\newtheorem*{corollary}{推论}
\newtheorem*{exercise}{习题}
\newtheorem*{example}{例}
\geometry{left=2.5cm,right=2.5cm,top=2.5cm,bottom=2.5cm}
\setCJKmainfont[BoldFont=Adobe Heiti Std R]{Adobe Song Std L}
\renewcommand{\today}{\number\year 年 \number\month 月 \number\day 日}
\newcommand{\D}{\displaystyle}\newcommand{\ri}{\Rightarrow}
\newcommand{\ds}{\displaystyle} \renewcommand{\ni}{\noindent}
\newcommand{\pa}{\partial} \newcommand{\Om}{\Omega}
\newcommand{\om}{\omega} \newcommand{\sik}{\sum_{i=1}^k}
\newcommand{\vov}{\Vert\omega\Vert} \newcommand{\Umy}{U_{\mu_i,y^i}}
\newcommand{\lamns}{\lambda_n^{^{\scriptstyle\sigma}}}
\newcommand{\chiomn}{\chi_{_{\Omega_n}}}
\newcommand{\ullim}{\underline{\lim}} \newcommand{\bsy}{\boldsymbol}
\newcommand{\mvb}{\mathversion{bold}} \newcommand{\la}{\lambda}
\newcommand{\La}{\Lambda} \newcommand{\va}{\varepsilon}
\newcommand{\be}{\beta} \newcommand{\al}{\alpha}
\newcommand{\dis}{\displaystyle} \newcommand{\R}{{\mathbb R}}
\newcommand{\N}{{\mathbb N}} \newcommand{\cF}{{\mathcal F}}
\newcommand{\gB}{{\mathfrak B}} \newcommand{\eps}{\epsilon}
\renewcommand\refname{参考文献}
\begin{document}
\title{\huge\red{\bf{为了浙大而奋斗:利用特征向量解三次方程}}} \author{\small{叶卢
    庆\footnote{叶卢庆(1992---),男,杭州师范大学理学院数学与应用数学专业
      本科在读,为了浙大而不懈奋斗!E-mail:h5411167@gmail.com}}\\{\small{杭州师范大学理学院,浙
      江~杭州~310036}}}
\maketitle
易得矩阵
$$
A=\begin{pmatrix}
  -p&-q&-r\\
1&0&0\\
0&1&0
\end{pmatrix}
$$
的特征方程为
$$
\lambda^3+p\lambda^2+q\lambda+r=0,
$$
其中 $p,q,r\in \mathbf{R}$.设特征方程有三个不同的实根
$\lambda_1,\lambda_2,\lambda_3$,对应的特征向量分别为
$\mathbf{v_1,v_2,v_3}$,也即,
$$
A(\mathbf{v_1})=\lambda_1\mathbf{v_1},A(\mathbf{v_2})=\lambda_2\mathbf{v_2},A(\mathbf{v_3})=\lambda_3\mathbf{v_3}.
$$
不妨设
$$
\mathbf{v_1}=(\lambda_1^2,\lambda_1,1),\mathbf{v_2}=(\lambda_2^2,\lambda_2,1),\mathbf{v_3}=(\lambda_3^2,\lambda_3,1).
$$
设
$$
\begin{cases}
  \alpha_1\mathbf{v_1}+\alpha_2\mathbf{v_2}+\alpha_3\mathbf{v_3}=(0,0,1),\\
  \beta_1\mathbf{v_1}+\beta_2\mathbf{v_2}+\beta_3\mathbf{v_3}=(0,1,0),\\
  \gamma_1\mathbf{v_1}+\gamma_2\mathbf{v_2}+\gamma_3\mathbf{v_3}=(1,0,0).\\
\end{cases}
$$

\end{document}







